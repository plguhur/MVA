
% Default to the notebook output style

    


% Inherit from the specified cell style.




    
\documentclass[11pt]{article}

    
    
    \usepackage[T1]{fontenc}
    % Nicer default font (+ math font) than Computer Modern for most use cases
    \usepackage{mathpazo}

    % Basic figure setup, for now with no caption control since it's done
    % automatically by Pandoc (which extracts ![](path) syntax from Markdown).
    \usepackage{graphicx}
    % We will generate all images so they have a width \maxwidth. This means
    % that they will get their normal width if they fit onto the page, but
    % are scaled down if they would overflow the margins.
    \makeatletter
    \def\maxwidth{\ifdim\Gin@nat@width>\linewidth\linewidth
    \else\Gin@nat@width\fi}
    \makeatother
    \let\Oldincludegraphics\includegraphics
    % Set max figure width to be 80% of text width, for now hardcoded.
    \renewcommand{\includegraphics}[1]{\Oldincludegraphics[width=.8\maxwidth]{#1}}
    % Ensure that by default, figures have no caption (until we provide a
    % proper Figure object with a Caption API and a way to capture that
    % in the conversion process - todo).
    \usepackage{caption}
    \DeclareCaptionLabelFormat{nolabel}{}
    \captionsetup{labelformat=nolabel}

    \usepackage{adjustbox} % Used to constrain images to a maximum size 
    \usepackage{xcolor} % Allow colors to be defined
    \usepackage{enumerate} % Needed for markdown enumerations to work
    \usepackage{geometry} % Used to adjust the document margins
    \usepackage{amsmath} % Equations
    \usepackage{amssymb} % Equations
    \usepackage{textcomp} % defines textquotesingle
    % Hack from http://tex.stackexchange.com/a/47451/13684:
    \AtBeginDocument{%
        \def\PYZsq{\textquotesingle}% Upright quotes in Pygmentized code
    }
    \usepackage{upquote} % Upright quotes for verbatim code
    \usepackage{eurosym} % defines \euro
    \usepackage[mathletters]{ucs} % Extended unicode (utf-8) support
    \usepackage[utf8x]{inputenc} % Allow utf-8 characters in the tex document
    \usepackage{fancyvrb} % verbatim replacement that allows latex
    \usepackage{grffile} % extends the file name processing of package graphics 
                         % to support a larger range 
    % The hyperref package gives us a pdf with properly built
    % internal navigation ('pdf bookmarks' for the table of contents,
    % internal cross-reference links, web links for URLs, etc.)
    \usepackage{hyperref}
    \usepackage{longtable} % longtable support required by pandoc >1.10
    \usepackage{booktabs}  % table support for pandoc > 1.12.2
    \usepackage[inline]{enumitem} % IRkernel/repr support (it uses the enumerate* environment)
    \usepackage[normalem]{ulem} % ulem is needed to support strikethroughs (\sout)
                                % normalem makes italics be italics, not underlines
    

    
    
    % Colors for the hyperref package
    \definecolor{urlcolor}{rgb}{0,.145,.698}
    \definecolor{linkcolor}{rgb}{.71,0.21,0.01}
    \definecolor{citecolor}{rgb}{.12,.54,.11}

    % ANSI colors
    \definecolor{ansi-black}{HTML}{3E424D}
    \definecolor{ansi-black-intense}{HTML}{282C36}
    \definecolor{ansi-red}{HTML}{E75C58}
    \definecolor{ansi-red-intense}{HTML}{B22B31}
    \definecolor{ansi-green}{HTML}{00A250}
    \definecolor{ansi-green-intense}{HTML}{007427}
    \definecolor{ansi-yellow}{HTML}{DDB62B}
    \definecolor{ansi-yellow-intense}{HTML}{B27D12}
    \definecolor{ansi-blue}{HTML}{208FFB}
    \definecolor{ansi-blue-intense}{HTML}{0065CA}
    \definecolor{ansi-magenta}{HTML}{D160C4}
    \definecolor{ansi-magenta-intense}{HTML}{A03196}
    \definecolor{ansi-cyan}{HTML}{60C6C8}
    \definecolor{ansi-cyan-intense}{HTML}{258F8F}
    \definecolor{ansi-white}{HTML}{C5C1B4}
    \definecolor{ansi-white-intense}{HTML}{A1A6B2}

    % commands and environments needed by pandoc snippets
    % extracted from the output of `pandoc -s`
    \providecommand{\tightlist}{%
      \setlength{\itemsep}{0pt}\setlength{\parskip}{0pt}}
    \DefineVerbatimEnvironment{Highlighting}{Verbatim}{commandchars=\\\{\}}
    % Add ',fontsize=\small' for more characters per line
    \newenvironment{Shaded}{}{}
    \newcommand{\KeywordTok}[1]{\textcolor[rgb]{0.00,0.44,0.13}{\textbf{{#1}}}}
    \newcommand{\DataTypeTok}[1]{\textcolor[rgb]{0.56,0.13,0.00}{{#1}}}
    \newcommand{\DecValTok}[1]{\textcolor[rgb]{0.25,0.63,0.44}{{#1}}}
    \newcommand{\BaseNTok}[1]{\textcolor[rgb]{0.25,0.63,0.44}{{#1}}}
    \newcommand{\FloatTok}[1]{\textcolor[rgb]{0.25,0.63,0.44}{{#1}}}
    \newcommand{\CharTok}[1]{\textcolor[rgb]{0.25,0.44,0.63}{{#1}}}
    \newcommand{\StringTok}[1]{\textcolor[rgb]{0.25,0.44,0.63}{{#1}}}
    \newcommand{\CommentTok}[1]{\textcolor[rgb]{0.38,0.63,0.69}{\textit{{#1}}}}
    \newcommand{\OtherTok}[1]{\textcolor[rgb]{0.00,0.44,0.13}{{#1}}}
    \newcommand{\AlertTok}[1]{\textcolor[rgb]{1.00,0.00,0.00}{\textbf{{#1}}}}
    \newcommand{\FunctionTok}[1]{\textcolor[rgb]{0.02,0.16,0.49}{{#1}}}
    \newcommand{\RegionMarkerTok}[1]{{#1}}
    \newcommand{\ErrorTok}[1]{\textcolor[rgb]{1.00,0.00,0.00}{\textbf{{#1}}}}
    \newcommand{\NormalTok}[1]{{#1}}
    
    % Additional commands for more recent versions of Pandoc
    \newcommand{\ConstantTok}[1]{\textcolor[rgb]{0.53,0.00,0.00}{{#1}}}
    \newcommand{\SpecialCharTok}[1]{\textcolor[rgb]{0.25,0.44,0.63}{{#1}}}
    \newcommand{\VerbatimStringTok}[1]{\textcolor[rgb]{0.25,0.44,0.63}{{#1}}}
    \newcommand{\SpecialStringTok}[1]{\textcolor[rgb]{0.73,0.40,0.53}{{#1}}}
    \newcommand{\ImportTok}[1]{{#1}}
    \newcommand{\DocumentationTok}[1]{\textcolor[rgb]{0.73,0.13,0.13}{\textit{{#1}}}}
    \newcommand{\AnnotationTok}[1]{\textcolor[rgb]{0.38,0.63,0.69}{\textbf{\textit{{#1}}}}}
    \newcommand{\CommentVarTok}[1]{\textcolor[rgb]{0.38,0.63,0.69}{\textbf{\textit{{#1}}}}}
    \newcommand{\VariableTok}[1]{\textcolor[rgb]{0.10,0.09,0.49}{{#1}}}
    \newcommand{\ControlFlowTok}[1]{\textcolor[rgb]{0.00,0.44,0.13}{\textbf{{#1}}}}
    \newcommand{\OperatorTok}[1]{\textcolor[rgb]{0.40,0.40,0.40}{{#1}}}
    \newcommand{\BuiltInTok}[1]{{#1}}
    \newcommand{\ExtensionTok}[1]{{#1}}
    \newcommand{\PreprocessorTok}[1]{\textcolor[rgb]{0.74,0.48,0.00}{{#1}}}
    \newcommand{\AttributeTok}[1]{\textcolor[rgb]{0.49,0.56,0.16}{{#1}}}
    \newcommand{\InformationTok}[1]{\textcolor[rgb]{0.38,0.63,0.69}{\textbf{\textit{{#1}}}}}
    \newcommand{\WarningTok}[1]{\textcolor[rgb]{0.38,0.63,0.69}{\textbf{\textit{{#1}}}}}
    
    
    % Define a nice break command that doesn't care if a line doesn't already
    % exist.
    \def\br{\hspace*{\fill} \\* }
    % Math Jax compatability definitions
    \def\gt{>}
    \def\lt{<}
    % Document parameters
    \title{A1\_GUHUR\_Pierre-Louis}
    
    
    

    % Pygments definitions
    
\makeatletter
\def\PY@reset{\let\PY@it=\relax \let\PY@bf=\relax%
    \let\PY@ul=\relax \let\PY@tc=\relax%
    \let\PY@bc=\relax \let\PY@ff=\relax}
\def\PY@tok#1{\csname PY@tok@#1\endcsname}
\def\PY@toks#1+{\ifx\relax#1\empty\else%
    \PY@tok{#1}\expandafter\PY@toks\fi}
\def\PY@do#1{\PY@bc{\PY@tc{\PY@ul{%
    \PY@it{\PY@bf{\PY@ff{#1}}}}}}}
\def\PY#1#2{\PY@reset\PY@toks#1+\relax+\PY@do{#2}}

\expandafter\def\csname PY@tok@gd\endcsname{\def\PY@tc##1{\textcolor[rgb]{0.63,0.00,0.00}{##1}}}
\expandafter\def\csname PY@tok@gu\endcsname{\let\PY@bf=\textbf\def\PY@tc##1{\textcolor[rgb]{0.50,0.00,0.50}{##1}}}
\expandafter\def\csname PY@tok@gt\endcsname{\def\PY@tc##1{\textcolor[rgb]{0.00,0.27,0.87}{##1}}}
\expandafter\def\csname PY@tok@gs\endcsname{\let\PY@bf=\textbf}
\expandafter\def\csname PY@tok@gr\endcsname{\def\PY@tc##1{\textcolor[rgb]{1.00,0.00,0.00}{##1}}}
\expandafter\def\csname PY@tok@cm\endcsname{\let\PY@it=\textit\def\PY@tc##1{\textcolor[rgb]{0.25,0.50,0.50}{##1}}}
\expandafter\def\csname PY@tok@vg\endcsname{\def\PY@tc##1{\textcolor[rgb]{0.10,0.09,0.49}{##1}}}
\expandafter\def\csname PY@tok@vi\endcsname{\def\PY@tc##1{\textcolor[rgb]{0.10,0.09,0.49}{##1}}}
\expandafter\def\csname PY@tok@vm\endcsname{\def\PY@tc##1{\textcolor[rgb]{0.10,0.09,0.49}{##1}}}
\expandafter\def\csname PY@tok@mh\endcsname{\def\PY@tc##1{\textcolor[rgb]{0.40,0.40,0.40}{##1}}}
\expandafter\def\csname PY@tok@cs\endcsname{\let\PY@it=\textit\def\PY@tc##1{\textcolor[rgb]{0.25,0.50,0.50}{##1}}}
\expandafter\def\csname PY@tok@ge\endcsname{\let\PY@it=\textit}
\expandafter\def\csname PY@tok@vc\endcsname{\def\PY@tc##1{\textcolor[rgb]{0.10,0.09,0.49}{##1}}}
\expandafter\def\csname PY@tok@il\endcsname{\def\PY@tc##1{\textcolor[rgb]{0.40,0.40,0.40}{##1}}}
\expandafter\def\csname PY@tok@go\endcsname{\def\PY@tc##1{\textcolor[rgb]{0.53,0.53,0.53}{##1}}}
\expandafter\def\csname PY@tok@cp\endcsname{\def\PY@tc##1{\textcolor[rgb]{0.74,0.48,0.00}{##1}}}
\expandafter\def\csname PY@tok@gi\endcsname{\def\PY@tc##1{\textcolor[rgb]{0.00,0.63,0.00}{##1}}}
\expandafter\def\csname PY@tok@gh\endcsname{\let\PY@bf=\textbf\def\PY@tc##1{\textcolor[rgb]{0.00,0.00,0.50}{##1}}}
\expandafter\def\csname PY@tok@ni\endcsname{\let\PY@bf=\textbf\def\PY@tc##1{\textcolor[rgb]{0.60,0.60,0.60}{##1}}}
\expandafter\def\csname PY@tok@nl\endcsname{\def\PY@tc##1{\textcolor[rgb]{0.63,0.63,0.00}{##1}}}
\expandafter\def\csname PY@tok@nn\endcsname{\let\PY@bf=\textbf\def\PY@tc##1{\textcolor[rgb]{0.00,0.00,1.00}{##1}}}
\expandafter\def\csname PY@tok@no\endcsname{\def\PY@tc##1{\textcolor[rgb]{0.53,0.00,0.00}{##1}}}
\expandafter\def\csname PY@tok@na\endcsname{\def\PY@tc##1{\textcolor[rgb]{0.49,0.56,0.16}{##1}}}
\expandafter\def\csname PY@tok@nb\endcsname{\def\PY@tc##1{\textcolor[rgb]{0.00,0.50,0.00}{##1}}}
\expandafter\def\csname PY@tok@nc\endcsname{\let\PY@bf=\textbf\def\PY@tc##1{\textcolor[rgb]{0.00,0.00,1.00}{##1}}}
\expandafter\def\csname PY@tok@nd\endcsname{\def\PY@tc##1{\textcolor[rgb]{0.67,0.13,1.00}{##1}}}
\expandafter\def\csname PY@tok@ne\endcsname{\let\PY@bf=\textbf\def\PY@tc##1{\textcolor[rgb]{0.82,0.25,0.23}{##1}}}
\expandafter\def\csname PY@tok@nf\endcsname{\def\PY@tc##1{\textcolor[rgb]{0.00,0.00,1.00}{##1}}}
\expandafter\def\csname PY@tok@si\endcsname{\let\PY@bf=\textbf\def\PY@tc##1{\textcolor[rgb]{0.73,0.40,0.53}{##1}}}
\expandafter\def\csname PY@tok@s2\endcsname{\def\PY@tc##1{\textcolor[rgb]{0.73,0.13,0.13}{##1}}}
\expandafter\def\csname PY@tok@nt\endcsname{\let\PY@bf=\textbf\def\PY@tc##1{\textcolor[rgb]{0.00,0.50,0.00}{##1}}}
\expandafter\def\csname PY@tok@nv\endcsname{\def\PY@tc##1{\textcolor[rgb]{0.10,0.09,0.49}{##1}}}
\expandafter\def\csname PY@tok@s1\endcsname{\def\PY@tc##1{\textcolor[rgb]{0.73,0.13,0.13}{##1}}}
\expandafter\def\csname PY@tok@dl\endcsname{\def\PY@tc##1{\textcolor[rgb]{0.73,0.13,0.13}{##1}}}
\expandafter\def\csname PY@tok@ch\endcsname{\let\PY@it=\textit\def\PY@tc##1{\textcolor[rgb]{0.25,0.50,0.50}{##1}}}
\expandafter\def\csname PY@tok@m\endcsname{\def\PY@tc##1{\textcolor[rgb]{0.40,0.40,0.40}{##1}}}
\expandafter\def\csname PY@tok@gp\endcsname{\let\PY@bf=\textbf\def\PY@tc##1{\textcolor[rgb]{0.00,0.00,0.50}{##1}}}
\expandafter\def\csname PY@tok@sh\endcsname{\def\PY@tc##1{\textcolor[rgb]{0.73,0.13,0.13}{##1}}}
\expandafter\def\csname PY@tok@ow\endcsname{\let\PY@bf=\textbf\def\PY@tc##1{\textcolor[rgb]{0.67,0.13,1.00}{##1}}}
\expandafter\def\csname PY@tok@sx\endcsname{\def\PY@tc##1{\textcolor[rgb]{0.00,0.50,0.00}{##1}}}
\expandafter\def\csname PY@tok@bp\endcsname{\def\PY@tc##1{\textcolor[rgb]{0.00,0.50,0.00}{##1}}}
\expandafter\def\csname PY@tok@c1\endcsname{\let\PY@it=\textit\def\PY@tc##1{\textcolor[rgb]{0.25,0.50,0.50}{##1}}}
\expandafter\def\csname PY@tok@fm\endcsname{\def\PY@tc##1{\textcolor[rgb]{0.00,0.00,1.00}{##1}}}
\expandafter\def\csname PY@tok@o\endcsname{\def\PY@tc##1{\textcolor[rgb]{0.40,0.40,0.40}{##1}}}
\expandafter\def\csname PY@tok@kc\endcsname{\let\PY@bf=\textbf\def\PY@tc##1{\textcolor[rgb]{0.00,0.50,0.00}{##1}}}
\expandafter\def\csname PY@tok@c\endcsname{\let\PY@it=\textit\def\PY@tc##1{\textcolor[rgb]{0.25,0.50,0.50}{##1}}}
\expandafter\def\csname PY@tok@mf\endcsname{\def\PY@tc##1{\textcolor[rgb]{0.40,0.40,0.40}{##1}}}
\expandafter\def\csname PY@tok@err\endcsname{\def\PY@bc##1{\setlength{\fboxsep}{0pt}\fcolorbox[rgb]{1.00,0.00,0.00}{1,1,1}{\strut ##1}}}
\expandafter\def\csname PY@tok@mb\endcsname{\def\PY@tc##1{\textcolor[rgb]{0.40,0.40,0.40}{##1}}}
\expandafter\def\csname PY@tok@ss\endcsname{\def\PY@tc##1{\textcolor[rgb]{0.10,0.09,0.49}{##1}}}
\expandafter\def\csname PY@tok@sr\endcsname{\def\PY@tc##1{\textcolor[rgb]{0.73,0.40,0.53}{##1}}}
\expandafter\def\csname PY@tok@mo\endcsname{\def\PY@tc##1{\textcolor[rgb]{0.40,0.40,0.40}{##1}}}
\expandafter\def\csname PY@tok@kd\endcsname{\let\PY@bf=\textbf\def\PY@tc##1{\textcolor[rgb]{0.00,0.50,0.00}{##1}}}
\expandafter\def\csname PY@tok@mi\endcsname{\def\PY@tc##1{\textcolor[rgb]{0.40,0.40,0.40}{##1}}}
\expandafter\def\csname PY@tok@kn\endcsname{\let\PY@bf=\textbf\def\PY@tc##1{\textcolor[rgb]{0.00,0.50,0.00}{##1}}}
\expandafter\def\csname PY@tok@cpf\endcsname{\let\PY@it=\textit\def\PY@tc##1{\textcolor[rgb]{0.25,0.50,0.50}{##1}}}
\expandafter\def\csname PY@tok@kr\endcsname{\let\PY@bf=\textbf\def\PY@tc##1{\textcolor[rgb]{0.00,0.50,0.00}{##1}}}
\expandafter\def\csname PY@tok@s\endcsname{\def\PY@tc##1{\textcolor[rgb]{0.73,0.13,0.13}{##1}}}
\expandafter\def\csname PY@tok@kp\endcsname{\def\PY@tc##1{\textcolor[rgb]{0.00,0.50,0.00}{##1}}}
\expandafter\def\csname PY@tok@w\endcsname{\def\PY@tc##1{\textcolor[rgb]{0.73,0.73,0.73}{##1}}}
\expandafter\def\csname PY@tok@kt\endcsname{\def\PY@tc##1{\textcolor[rgb]{0.69,0.00,0.25}{##1}}}
\expandafter\def\csname PY@tok@sc\endcsname{\def\PY@tc##1{\textcolor[rgb]{0.73,0.13,0.13}{##1}}}
\expandafter\def\csname PY@tok@sb\endcsname{\def\PY@tc##1{\textcolor[rgb]{0.73,0.13,0.13}{##1}}}
\expandafter\def\csname PY@tok@sa\endcsname{\def\PY@tc##1{\textcolor[rgb]{0.73,0.13,0.13}{##1}}}
\expandafter\def\csname PY@tok@k\endcsname{\let\PY@bf=\textbf\def\PY@tc##1{\textcolor[rgb]{0.00,0.50,0.00}{##1}}}
\expandafter\def\csname PY@tok@se\endcsname{\let\PY@bf=\textbf\def\PY@tc##1{\textcolor[rgb]{0.73,0.40,0.13}{##1}}}
\expandafter\def\csname PY@tok@sd\endcsname{\let\PY@it=\textit\def\PY@tc##1{\textcolor[rgb]{0.73,0.13,0.13}{##1}}}

\def\PYZbs{\char`\\}
\def\PYZus{\char`\_}
\def\PYZob{\char`\{}
\def\PYZcb{\char`\}}
\def\PYZca{\char`\^}
\def\PYZam{\char`\&}
\def\PYZlt{\char`\<}
\def\PYZgt{\char`\>}
\def\PYZsh{\char`\#}
\def\PYZpc{\char`\%}
\def\PYZdl{\char`\$}
\def\PYZhy{\char`\-}
\def\PYZsq{\char`\'}
\def\PYZdq{\char`\"}
\def\PYZti{\char`\~}
% for compatibility with earlier versions
\def\PYZat{@}
\def\PYZlb{[}
\def\PYZrb{]}
\makeatother


    % Exact colors from NB
    \definecolor{incolor}{rgb}{0.0, 0.0, 0.5}
    \definecolor{outcolor}{rgb}{0.545, 0.0, 0.0}



    
    % Prevent overflowing lines due to hard-to-break entities
    \sloppy 
    % Setup hyperref package
    \hypersetup{
      breaklinks=true,  % so long urls are correctly broken across lines
      colorlinks=true,
      urlcolor=urlcolor,
      linkcolor=linkcolor,
      citecolor=citecolor,
      }
    % Slightly bigger margins than the latex defaults
    
    \geometry{verbose,tmargin=1in,bmargin=1in,lmargin=1in,rmargin=1in}
    
    

    \begin{document}
    
    
    \maketitle
    
    

    
    Object recognition and computer vision 2018/2019

Jean Ponce, Ivan Laptev, Cordelia Schmid and Josef Sivic

Assignment 1: Instance-level recognition

Adapted from practicals from Andrea Vedaldi and Andrew Zisserman by Gul
Varol and Ignacio Rocco

    \textbf{STUDENT}: Pierre-Louis Guhur

\textbf{EMAIL}: pierre-louis.guhur@ens-paris-saclay.fr

    \hypertarget{guidelines}{%
\section{Guidelines}\label{guidelines}}

The purpose of this assignment is that you get hands-on experience with
the topics covered in class, which will help you understand these topics
better. Therefore, ** it is imperative that you do this assignment
yourself. No code sharing will be tolerated. **

Once you have completed the assignment, you will submit the
\texttt{ipynb} file containing \textbf{both} code and results. For this,
make sure to \textbf{run your notebook completely before submitting}.

The \texttt{ipynb} must be named using the following format:
\textbf{A1\_LASTNAME\_Firstname.ipynb}, and submitted in the
\textbf{class Moodle page}.

    \hypertarget{goal}{%
\section{Goal}\label{goal}}

The goal of instance-level recognition is to match (recognize) a
specific object or scene. Examples include recognizing a specific
building, such as Notre Dame, or a specific painting, such as `Starry
Night' by Van Gogh. The object is recognized despite changes in scale,
camera viewpoint, illumination conditions and partial occlusion. An
important application is image retrieval - starting from an image of an
object of interest (the query), search through an image dataset to
obtain (or retrieve) those images that contain the target object.

The goal of this assignment is to experiment and get basic practical
experience with the methods that enable specific object recognition. It
includes: (i) using SIFT features to obtain sparse matches between two
images; (ii) using similarity co-variant detectors to cover changes in
viewpoint; (iii) vector quantizing the SIFT descriptors into visual
words to enable large scale retrieval; and (iv) constructing and using
an image retrieval system to identify objects.

    \begin{Verbatim}[commandchars=\\\{\}]
{\color{incolor}In [{\color{incolor}1}]:} \PY{k+kn}{import} \PY{n+nn}{cyvlfeat}
        \PY{k+kn}{import} \PY{n+nn}{numpy} \PY{k}{as} \PY{n+nn}{np}
        \PY{k+kn}{from} \PY{n+nn}{scipy}\PY{n+nn}{.}\PY{n+nn}{misc} \PY{k}{import} \PY{n}{imread}\PY{p}{,}\PY{n}{imresize}\PY{p}{,}\PY{n}{imrotate}
        \PY{k+kn}{from} \PY{n+nn}{urllib}\PY{n+nn}{.}\PY{n+nn}{request} \PY{k}{import} \PY{n}{urlopen}
        \PY{k+kn}{import} \PY{n+nn}{matplotlib} \PY{k}{as} \PY{n+nn}{mpl}
        \PY{k+kn}{import} \PY{n+nn}{matplotlib}\PY{n+nn}{.}\PY{n+nn}{pyplot} \PY{k}{as} \PY{n+nn}{plt}
        \PY{k+kn}{import} \PY{n+nn}{warnings}
        \PY{k+kn}{from} \PY{n+nn}{cyvlfeat}\PY{n+nn}{.}\PY{n+nn}{plot} \PY{k}{import} \PY{n}{plotframes}
        \PY{k+kn}{from} \PY{n+nn}{scipy}\PY{n+nn}{.}\PY{n+nn}{io} \PY{k}{import} \PY{n}{loadmat}
        \PY{k+kn}{import} \PY{n+nn}{numpy} \PY{k}{as} \PY{n+nn}{np}
        
        \PY{c+c1}{\PYZsh{} change some default matplotlib parameters}
        \PY{n}{mpl}\PY{o}{.}\PY{n}{rcParams}\PY{p}{[}\PY{l+s+s1}{\PYZsq{}}\PY{l+s+s1}{axes.grid}\PY{l+s+s1}{\PYZsq{}}\PY{p}{]} \PY{o}{=} \PY{k+kc}{False}
        \PY{n}{mpl}\PY{o}{.}\PY{n}{rcParams}\PY{p}{[}\PY{l+s+s1}{\PYZsq{}}\PY{l+s+s1}{figure.dpi}\PY{l+s+s1}{\PYZsq{}}\PY{p}{]} \PY{o}{=} \PY{l+m+mi}{120}
        
        \PY{c+c1}{\PYZsh{} ignore warnings}
        \PY{n}{warnings}\PY{o}{.}\PY{n}{filterwarnings}\PY{p}{(}\PY{l+s+s1}{\PYZsq{}}\PY{l+s+s1}{ignore}\PY{l+s+s1}{\PYZsq{}}\PY{p}{)}
        
        \PY{k}{def} \PY{n+nf}{rgb2gray}\PY{p}{(}\PY{n}{rgb}\PY{p}{)}\PY{p}{:}
            \PY{k}{return} \PY{n}{np}\PY{o}{.}\PY{n}{float32}\PY{p}{(}\PY{n}{np}\PY{o}{.}\PY{n}{dot}\PY{p}{(}\PY{n}{rgb}\PY{p}{[}\PY{o}{.}\PY{o}{.}\PY{o}{.}\PY{p}{,}\PY{p}{:}\PY{l+m+mi}{3}\PY{p}{]}\PY{p}{,} \PY{p}{[}\PY{l+m+mf}{0.299}\PY{p}{,} \PY{l+m+mf}{0.587}\PY{p}{,} \PY{l+m+mf}{0.114}\PY{p}{]}\PY{p}{)}\PY{o}{/}\PY{l+m+mf}{255.0}\PY{p}{)}
\end{Verbatim}

    \hypertarget{part-1-sparse-features-for-matching-specific-objects-in-images}{%
\section{Part 1: Sparse features for matching specific objects in
images}\label{part-1-sparse-features-for-matching-specific-objects-in-images}}

    \hypertarget{feature-point-detection}{%
\subsection{Feature point detection}\label{feature-point-detection}}

The SIFT feature has both a \emph{detector} and a \emph{descriptor}. The
\emph{detector} used in SIFT corresponds to the ``difference of
Gaussians'' (DoG) detector, which is an approximation of the ``Laplacian
of Gaussian'' (LoG) detector.

We will start by computing and visualizing the SIFT feature detections
(usually called frames) for two images of the same object (a building
facade). Load an image, rotate and scale it, and then display the
original and transformed pair:

    \begin{Verbatim}[commandchars=\\\{\}]
{\color{incolor}In [{\color{incolor}79}]:} \PY{n}{im1} \PY{o}{=} \PY{n}{imread}\PY{p}{(}\PY{l+s+s1}{\PYZsq{}}\PY{l+s+s1}{data/oxbuild\PYZus{}lite/all\PYZus{}souls\PYZus{}000002.jpg}\PY{l+s+s1}{\PYZsq{}}\PY{p}{)}
         \PY{c+c1}{\PYZsh{} Let the second image be a rotated and scaled version of the first}
         \PY{n}{im1prime} \PY{o}{=} \PY{n}{imresize}\PY{p}{(}\PY{n}{imrotate}\PY{p}{(}\PY{n}{np}\PY{o}{.}\PY{n}{pad}\PY{p}{(}\PY{n}{im1}\PY{p}{,}\PY{p}{(}\PY{p}{(}\PY{l+m+mi}{0}\PY{p}{,}\PY{l+m+mi}{0}\PY{p}{)}\PY{p}{,}\PY{p}{(}\PY{l+m+mi}{200}\PY{p}{,}\PY{l+m+mi}{200}\PY{p}{)}\PY{p}{,}\PY{p}{(}\PY{l+m+mi}{0}\PY{p}{,}\PY{l+m+mi}{0}\PY{p}{)}\PY{p}{)}\PY{p}{,}\PY{l+s+s1}{\PYZsq{}}\PY{l+s+s1}{constant}\PY{l+s+s1}{\PYZsq{}}\PY{p}{)}\PY{p}{,}\PY{l+m+mi}{35}\PY{p}{,}\PY{l+s+s1}{\PYZsq{}}\PY{l+s+s1}{bilinear}\PY{l+s+s1}{\PYZsq{}}\PY{p}{)}\PY{p}{,}\PY{l+m+mf}{0.7}\PY{p}{)}
         
         \PY{n}{f}\PY{p}{,} \PY{p}{(}\PY{n}{ax1}\PY{p}{,} \PY{n}{ax2}\PY{p}{)} \PY{o}{=} \PY{n}{plt}\PY{o}{.}\PY{n}{subplots}\PY{p}{(}\PY{l+m+mi}{1}\PY{p}{,} \PY{l+m+mi}{2}\PY{p}{)}
         \PY{n}{ax1}\PY{o}{.}\PY{n}{imshow}\PY{p}{(}\PY{n}{im1}\PY{p}{)}
         \PY{n}{ax2}\PY{o}{.}\PY{n}{imshow}\PY{p}{(}\PY{n}{im1prime}\PY{p}{)}
         \PY{n}{plt}\PY{o}{.}\PY{n}{show}\PY{p}{(}\PY{p}{)}
\end{Verbatim}

    \begin{center}
    \adjustimage{max size={0.9\linewidth}{0.9\paperheight}}{output_7_0.png}
    \end{center}
    { \hspace*{\fill} \\}
    
    A SIFT frame is a circle with an orientation and is specified by four
parameters: the center \(x\), \(y\), the scale \(s\), and the rotation
\(\theta\) (in radians), resulting in a vector of four parameters
\((x, y, s, \theta)\).

Now compute and visualise the SIFT feature detections (frames):

    \begin{Verbatim}[commandchars=\\\{\}]
{\color{incolor}In [{\color{incolor}5}]:} \PY{p}{[}\PY{n}{frames1}\PY{p}{,} \PY{n}{descrs1}\PY{p}{]} \PY{o}{=} \PY{n}{cyvlfeat}\PY{o}{.}\PY{n}{sift}\PY{o}{.}\PY{n}{sift}\PY{p}{(}\PY{n}{rgb2gray}\PY{p}{(}\PY{n}{im1}\PY{p}{)}\PY{p}{,}\PY{n}{peak\PYZus{}thresh}\PY{o}{=}\PY{l+m+mf}{0.01}\PY{p}{)}
        
        \PY{p}{[}\PY{n}{frames1prime}\PY{p}{,} \PY{n}{descrs1prime}\PY{p}{]} \PY{o}{=} \PY{n}{cyvlfeat}\PY{o}{.}\PY{n}{sift}\PY{o}{.}\PY{n}{sift}\PY{p}{(}\PY{n}{rgb2gray}\PY{p}{(}\PY{n}{im1prime}\PY{p}{)}\PY{p}{,}\PY{n}{peak\PYZus{}thresh}\PY{o}{=}\PY{l+m+mf}{0.01}\PY{p}{)}
        
        \PY{n}{f}\PY{p}{,} \PY{p}{(}\PY{n}{ax1}\PY{p}{,} \PY{n}{ax2}\PY{p}{)} \PY{o}{=} \PY{n}{plt}\PY{o}{.}\PY{n}{subplots}\PY{p}{(}\PY{l+m+mi}{1}\PY{p}{,} \PY{l+m+mi}{2}\PY{p}{)}
        \PY{n}{plt}\PY{o}{.}\PY{n}{sca}\PY{p}{(}\PY{n}{ax1}\PY{p}{)}
        \PY{n}{plt}\PY{o}{.}\PY{n}{imshow}\PY{p}{(}\PY{n}{im1}\PY{p}{)}
        \PY{n}{plotframes}\PY{p}{(}\PY{n}{frames1}\PY{p}{,}\PY{n}{linewidth}\PY{o}{=}\PY{l+m+mi}{1}\PY{p}{)}
        
        \PY{n}{plt}\PY{o}{.}\PY{n}{sca}\PY{p}{(}\PY{n}{ax2}\PY{p}{)}
        \PY{n}{plt}\PY{o}{.}\PY{n}{imshow}\PY{p}{(}\PY{n}{im1prime}\PY{p}{)}
        \PY{n}{plotframes}\PY{p}{(}\PY{n}{frames1prime}\PY{p}{,}\PY{n}{linewidth}\PY{o}{=}\PY{l+m+mi}{1}\PY{p}{)}
        
        \PY{n}{plt}\PY{o}{.}\PY{n}{show}\PY{p}{(}\PY{p}{)}
\end{Verbatim}

    \begin{center}
    \adjustimage{max size={0.9\linewidth}{0.9\paperheight}}{output_9_0.png}
    \end{center}
    { \hspace*{\fill} \\}
    
    Examine the second image and its rotated and scaled version and convince
yourself that the detections overlap the same scene regions (even though
the circles' positions have moved in the image and their radius' have
changed). This demonstrates that the detection process (is co-variant or
equi-variant) with translations, rotations and isotropic scalings. This
class of transformations is known as a similarity or equiform.

    \hypertarget{task-1.1}{%
\subsubsection{:: TASK 1.1 ::}\label{task-1.1}}

Now repeat the exercise with a pair of natural images.

Start by loading the second one:
\texttt{data/oxbuild\_lite/all\_souls\_000015.jpg}

Plot the images and feature frames. Again you should see that many of
the detections overlap the same scene region. Note that, while
repeatability occurs for the pair of natural views, it is much better
for the synthetically rotated pair.

    \begin{Verbatim}[commandchars=\\\{\}]
{\color{incolor}In [{\color{incolor}80}]:} \PY{n}{im2} \PY{o}{=} \PY{n}{imread}\PY{p}{(}\PY{l+s+s1}{\PYZsq{}}\PY{l+s+s1}{data/oxbuild\PYZus{}lite/all\PYZus{}souls\PYZus{}000015.jpg}\PY{l+s+s1}{\PYZsq{}}\PY{p}{)}
         \PY{p}{[}\PY{n}{frames2}\PY{p}{,} \PY{n}{descrs2}\PY{p}{]} \PY{o}{=} \PY{n}{cyvlfeat}\PY{o}{.}\PY{n}{sift}\PY{o}{.}\PY{n}{sift}\PY{p}{(}\PY{n}{rgb2gray}\PY{p}{(}\PY{n}{im2}\PY{p}{)}\PY{p}{,}\PY{n}{peak\PYZus{}thresh}\PY{o}{=}\PY{l+m+mf}{0.01}\PY{p}{)}
         
         \PY{n}{f}\PY{p}{,} \PY{p}{(}\PY{n}{ax1}\PY{p}{,} \PY{n}{ax2}\PY{p}{,} \PY{n}{ax3}\PY{p}{)} \PY{o}{=} \PY{n}{plt}\PY{o}{.}\PY{n}{subplots}\PY{p}{(}\PY{l+m+mi}{1}\PY{p}{,} \PY{l+m+mi}{3}\PY{p}{)}
         \PY{n}{plt}\PY{o}{.}\PY{n}{sca}\PY{p}{(}\PY{n}{ax1}\PY{p}{)}
         \PY{n}{plt}\PY{o}{.}\PY{n}{imshow}\PY{p}{(}\PY{n}{im1}\PY{p}{)}
         \PY{n}{plotframes}\PY{p}{(}\PY{n}{frames1}\PY{p}{,}\PY{n}{linewidth}\PY{o}{=}\PY{l+m+mi}{1}\PY{p}{)}
         
         \PY{n}{plt}\PY{o}{.}\PY{n}{sca}\PY{p}{(}\PY{n}{ax2}\PY{p}{)}
         \PY{n}{plt}\PY{o}{.}\PY{n}{imshow}\PY{p}{(}\PY{n}{im1prime}\PY{p}{)}
         \PY{n}{plotframes}\PY{p}{(}\PY{n}{frames1prime}\PY{p}{,}\PY{n}{linewidth}\PY{o}{=}\PY{l+m+mi}{1}\PY{p}{)}
         
         \PY{n}{plt}\PY{o}{.}\PY{n}{sca}\PY{p}{(}\PY{n}{ax3}\PY{p}{)}
         \PY{n}{plt}\PY{o}{.}\PY{n}{imshow}\PY{p}{(}\PY{n}{im2}\PY{p}{)}
         \PY{n}{plotframes}\PY{p}{(}\PY{n}{frames2}\PY{p}{,}\PY{n}{linewidth}\PY{o}{=}\PY{l+m+mi}{1}\PY{p}{)}
         
         \PY{n}{plt}\PY{o}{.}\PY{n}{show}\PY{p}{(}\PY{p}{)}
\end{Verbatim}

    \begin{center}
    \adjustimage{max size={0.9\linewidth}{0.9\paperheight}}{output_12_0.png}
    \end{center}
    { \hspace*{\fill} \\}
    
    The number of detected features can be controlled by changing the
peak\_thresh option. A larger value will select features that correspond
to higher contrast structures in the image.

    \hypertarget{task-1.2}{%
\subsubsection{:: TASK 1.2 ::}\label{task-1.2}}

\textbf{\emph{For the same image, produce 3 sub-figures with different
values of peak\_thresh. Comment.}}

    \begin{Verbatim}[commandchars=\\\{\}]
{\color{incolor}In [{\color{incolor}32}]:} \PY{n}{thresholds} \PY{o}{=} \PY{p}{[}\PY{l+m+mf}{0.001}\PY{p}{,} \PY{l+m+mf}{0.01}\PY{p}{,} \PY{l+m+mf}{0.05}\PY{p}{]}
         \PY{n}{plt}\PY{o}{.}\PY{n}{figure}\PY{p}{(}\PY{n}{figsize}\PY{o}{=}\PY{p}{(}\PY{l+m+mi}{10}\PY{p}{,}\PY{l+m+mi}{20}\PY{p}{)}\PY{p}{)}
         
         \PY{k}{for} \PY{n}{i}\PY{p}{,} \PY{n}{thresh} \PY{o+ow}{in} \PY{n+nb}{enumerate}\PY{p}{(}\PY{n}{thresholds}\PY{p}{)}\PY{p}{:}
             \PY{p}{[}\PY{n}{frames}\PY{p}{,} \PY{n}{descrs}\PY{p}{]} \PY{o}{=} \PY{n}{cyvlfeat}\PY{o}{.}\PY{n}{sift}\PY{o}{.}\PY{n}{sift}\PY{p}{(}\PY{n}{rgb2gray}\PY{p}{(}\PY{n}{im1}\PY{p}{)}\PY{p}{,}\PY{n}{peak\PYZus{}thresh}\PY{o}{=}\PY{n}{thresh}\PY{p}{)}
             \PY{n}{ax} \PY{o}{=} \PY{n}{plt}\PY{o}{.}\PY{n}{subplot}\PY{p}{(}\PY{l+m+mi}{1}\PY{p}{,} \PY{n+nb}{len}\PY{p}{(}\PY{n}{thresholds}\PY{p}{)}\PY{p}{,} \PY{n}{i}\PY{o}{+}\PY{l+m+mi}{1}\PY{p}{)}
             \PY{n}{ax}\PY{o}{.}\PY{n}{tick\PYZus{}params}\PY{p}{(}\PY{n}{labelbottom}\PY{o}{=}\PY{k+kc}{False}\PY{p}{,} \PY{n}{labelleft}\PY{o}{=}\PY{k+kc}{False}\PY{p}{)}  
             \PY{n}{plt}\PY{o}{.}\PY{n}{imshow}\PY{p}{(}\PY{n}{im1}\PY{p}{)}
             \PY{n}{plt}\PY{o}{.}\PY{n}{title}\PY{p}{(}\PY{l+s+s2}{\PYZdq{}}\PY{l+s+s2}{peak\PYZus{}thresh = }\PY{l+s+si}{\PYZpc{}.3f}\PY{l+s+s2}{\PYZdq{}} \PY{o}{\PYZpc{}} \PY{n}{thresh}\PY{p}{)}
             \PY{n}{plotframes}\PY{p}{(}\PY{n}{frames}\PY{p}{,}\PY{n}{linewidth}\PY{o}{=}\PY{l+m+mi}{1}\PY{p}{)}
\end{Verbatim}

    \begin{center}
    \adjustimage{max size={0.9\linewidth}{0.9\paperheight}}{output_15_0.png}
    \end{center}
    { \hspace*{\fill} \\}
    
    The \texttt{peak\_thresh} parameter is responsable for granularity
degree of feature detections. With a too small value, so many features
are detected that some noise points are considered -- which will keep
the algorithm from detecting real pattern. At the opposite, a too strong
value provides few indications to find the correspondance between two
images.

The parameter corresponds to the minimal value of the DoG to be
considered as a feature. On the DoG, peaks are the regions with an high
contrast. This explains why a too small value of \texttt{peak\_thresh}
provides a large number of features.

    \hypertarget{feature-point-description-and-matching}{%
\subsection{Feature point description and
matching}\label{feature-point-description-and-matching}}

    \hypertarget{introduction-to-feature-descriptors}{%
\subsubsection{Introduction to feature
descriptors}\label{introduction-to-feature-descriptors}}

The parameters \((t_x, t_y, s, \theta)\) of the detected \emph{frames},
can be used to extract a \textbf{scaled and oriented RGB patch} around
\((t_x,t_y)\), used to \emph{describe} the feature point.

The simplest possible descriptor would be to (i) resize these patches to
a common size (eg. 30x30) and (ii) flatten to a vector. However, in
practice we use more sophisticated descriptors such as SIFT, that is
based on histograms of gradient orientations.

    \hypertarget{task-1.3}{%
\subsubsection{:: TASK 1.3 ::}\label{task-1.3}}

\textbf{\emph{What is the interest of using SIFT descriptors over these
flattened RGB patches?}}

    RGB patches are highly dependant on the illumination of the scene. By
using the gradients of patches, SIFT descriptors are using instead the
structure of the image.

    \hypertarget{descriptor-matching}{%
\subsubsection{Descriptor matching}\label{descriptor-matching}}

SIFT descriptors are 128-dimensional vectors, and can be directly
\emph{matched} to find correspondences between images. We will start
with the simplest matching scheme (first nearest neighbour of
descriptors in terms of Euclidean distance) and then add more
sophisticated methods to eliminate any mismatches.

    \hypertarget{task-1.4}{%
\subsubsection{:: TASK 1.4 ::}\label{task-1.4}}

For each descriptor in im1, assign a matching descriptor in im2 by
picking its first nearest neighbour.

Populate the second column of the matches vector.

    \begin{Verbatim}[commandchars=\\\{\}]
{\color{incolor}In [{\color{incolor}86}]:} \PY{p}{[}\PY{n}{frames1}\PY{p}{,} \PY{n}{descrs1}\PY{p}{]} \PY{o}{=} \PY{n}{cyvlfeat}\PY{o}{.}\PY{n}{sift}\PY{o}{.}\PY{n}{sift}\PY{p}{(}\PY{n}{rgb2gray}\PY{p}{(}\PY{n}{im1}\PY{p}{)}\PY{p}{,}\PY{n}{peak\PYZus{}thresh}\PY{o}{=}\PY{l+m+mf}{0.01}\PY{p}{)}
         \PY{p}{[}\PY{n}{frames2}\PY{p}{,} \PY{n}{descrs2}\PY{p}{]} \PY{o}{=} \PY{n}{cyvlfeat}\PY{o}{.}\PY{n}{sift}\PY{o}{.}\PY{n}{sift}\PY{p}{(}\PY{n}{rgb2gray}\PY{p}{(}\PY{n}{im2}\PY{p}{)}\PY{p}{,}\PY{n}{peak\PYZus{}thresh}\PY{o}{=}\PY{l+m+mf}{0.01}\PY{p}{)}
\end{Verbatim}

    \begin{Verbatim}[commandchars=\\\{\}]
{\color{incolor}In [{\color{incolor}105}]:} \PY{n}{N\PYZus{}frames1} \PY{o}{=} \PY{n}{frames1}\PY{o}{.}\PY{n}{shape}\PY{p}{[}\PY{l+m+mi}{0}\PY{p}{]}
          \PY{n}{N\PYZus{}frames2} \PY{o}{=} \PY{n}{frames2}\PY{o}{.}\PY{n}{shape}\PY{p}{[}\PY{l+m+mi}{0}\PY{p}{]}
          
          \PY{n}{matches}\PY{o}{=}\PY{n}{np}\PY{o}{.}\PY{n}{zeros}\PY{p}{(}\PY{p}{(}\PY{n}{N\PYZus{}frames1}\PY{p}{,}\PY{l+m+mi}{2}\PY{p}{)}\PY{p}{,} \PY{n}{dtype}\PY{o}{=}\PY{n}{np}\PY{o}{.}\PY{n}{int}\PY{p}{)}
          \PY{c+c1}{\PYZsh{} the first column of the matrix are the indices on image 1: 0,1,2,....,N\PYZus{}frames1\PYZhy{}1}
          \PY{n}{matches}\PY{p}{[}\PY{p}{:}\PY{p}{,}\PY{l+m+mi}{0}\PY{p}{]} \PY{o}{=} \PY{n+nb}{range}\PY{p}{(}\PY{n}{N\PYZus{}frames1}\PY{p}{)}
          \PY{c+c1}{\PYZsh{} the second are the indices of frames closest on image 2 (matches)}
          \PY{n}{matches}\PY{p}{[}\PY{p}{:}\PY{p}{,}\PY{l+m+mi}{1}\PY{p}{]} \PY{o}{=} \PY{p}{[}\PY{n}{np}\PY{o}{.}\PY{n}{argmin}\PY{p}{(}\PY{p}{[}\PY{n}{np}\PY{o}{.}\PY{n}{sum}\PY{p}{(}\PY{p}{(}\PY{n}{desc1} \PY{o}{\PYZhy{}} \PY{n}{desc2}\PY{p}{)}\PY{o}{*}\PY{o}{*}\PY{l+m+mi}{2}\PY{p}{)} \PY{k}{for} \PY{n}{desc2} \PY{o+ow}{in} \PY{n}{descrs2}\PY{p}{]}\PY{p}{)} \PY{k}{for} \PY{n}{desc1} \PY{o+ow}{in} \PY{n}{descrs1}\PY{p}{]} 
\end{Verbatim}

    \begin{Verbatim}[commandchars=\\\{\}]
{\color{incolor}In [{\color{incolor}106}]:} \PY{c+c1}{\PYZsh{} plot matches}
          \PY{n}{plt}\PY{o}{.}\PY{n}{imshow}\PY{p}{(}\PY{n}{np}\PY{o}{.}\PY{n}{concatenate}\PY{p}{(}\PY{p}{(}\PY{n}{im1}\PY{p}{,}\PY{n}{im2}\PY{p}{)}\PY{p}{,}\PY{n}{axis}\PY{o}{=}\PY{l+m+mi}{1}\PY{p}{)}\PY{p}{)}
          \PY{k}{for} \PY{n}{i} \PY{o+ow}{in} \PY{n+nb}{range}\PY{p}{(}\PY{n}{N\PYZus{}frames1}\PY{p}{)}\PY{p}{:}
              \PY{n}{j}\PY{o}{=}\PY{n}{matches}\PY{p}{[}\PY{n}{i}\PY{p}{,}\PY{l+m+mi}{1}\PY{p}{]}
              \PY{c+c1}{\PYZsh{} plot dots at feature positions}
              \PY{n}{plt}\PY{o}{.}\PY{n}{gca}\PY{p}{(}\PY{p}{)}\PY{o}{.}\PY{n}{scatter}\PY{p}{(}\PY{p}{[}\PY{n}{frames1}\PY{p}{[}\PY{n}{i}\PY{p}{,}\PY{l+m+mi}{0}\PY{p}{]}\PY{p}{,}\PY{n}{im1}\PY{o}{.}\PY{n}{shape}\PY{p}{[}\PY{l+m+mi}{1}\PY{p}{]}\PY{o}{+}\PY{n}{frames2}\PY{p}{[}\PY{n}{j}\PY{p}{,}\PY{l+m+mi}{0}\PY{p}{]}\PY{p}{]}\PY{p}{,} \PY{p}{[}\PY{n}{frames1}\PY{p}{[}\PY{n}{i}\PY{p}{,}\PY{l+m+mi}{1}\PY{p}{]}\PY{p}{,}\PY{n}{frames2}\PY{p}{[}\PY{n}{j}\PY{p}{,}\PY{l+m+mi}{1}\PY{p}{]}\PY{p}{]}\PY{p}{,} \PY{n}{s}\PY{o}{=}\PY{l+m+mi}{5}\PY{p}{,} \PY{n}{c}\PY{o}{=}\PY{p}{[}\PY{l+m+mi}{0}\PY{p}{,}\PY{l+m+mi}{1}\PY{p}{,}\PY{l+m+mi}{0}\PY{p}{]}\PY{p}{)}
              \PY{c+c1}{\PYZsh{} plot lines}
              \PY{n}{plt}\PY{o}{.}\PY{n}{plot}\PY{p}{(}\PY{p}{[}\PY{n}{frames1}\PY{p}{[}\PY{n}{i}\PY{p}{,}\PY{l+m+mi}{0}\PY{p}{]}\PY{p}{,}\PY{n}{im1}\PY{o}{.}\PY{n}{shape}\PY{p}{[}\PY{l+m+mi}{1}\PY{p}{]}\PY{o}{+}\PY{n}{frames2}\PY{p}{[}\PY{n}{j}\PY{p}{,}\PY{l+m+mi}{0}\PY{p}{]}\PY{p}{]}\PY{p}{,}\PY{p}{[}\PY{n}{frames1}\PY{p}{[}\PY{n}{i}\PY{p}{,}\PY{l+m+mi}{1}\PY{p}{]}\PY{p}{,}\PY{n}{frames2}\PY{p}{[}\PY{n}{j}\PY{p}{,}\PY{l+m+mi}{1}\PY{p}{]}\PY{p}{]}\PY{p}{,}\PY{n}{linewidth}\PY{o}{=}\PY{l+m+mf}{0.5}\PY{p}{)}
          \PY{n}{plt}\PY{o}{.}\PY{n}{show}\PY{p}{(}\PY{p}{)}
\end{Verbatim}

    \begin{center}
    \adjustimage{max size={0.9\linewidth}{0.9\paperheight}}{output_25_0.png}
    \end{center}
    { \hspace*{\fill} \\}
    
    \hypertarget{improving-sift-matching-i-using-lowes-second-nearest-neighbour-test}{%
\subsection{Improving SIFT matching (i) using Lowe's second nearest
neighbour
test}\label{improving-sift-matching-i-using-lowes-second-nearest-neighbour-test}}

    Lowe introduced a second nearest neighbour (2nd NN) test to identify,
and hence remove, ambiguous matches. The idea is to identify distinctive
matches by a threshold on the ratio of first to second NN distances.

The ratio is:
\[NN_{ratio} = \frac{1^{st}\text{NN distance}}{2^{nd}\text{NN distance}}.\]

    \hypertarget{task-1.5}{%
\subsubsection{:: TASK 1.5 ::}\label{task-1.5}}

\textbf{\emph{For each descriptor in im1, compute the ratio between the
first and second nearest neighbour distances.}}

\textbf{\emph{Populate the ratio vector.}}

    \begin{Verbatim}[commandchars=\\\{\}]
{\color{incolor}In [{\color{incolor}107}]:} \PY{k}{def} \PY{n+nf}{range2}\PY{p}{(}\PY{n}{N}\PY{p}{,} \PY{n}{x}\PY{p}{)}\PY{p}{:}
              \PY{l+s+sd}{\PYZdq{}\PYZdq{}\PYZdq{} Return the range from 0 to N without x\PYZdq{}\PYZdq{}\PYZdq{}}
              \PY{k}{for} \PY{n}{i} \PY{o+ow}{in} \PY{n+nb}{range}\PY{p}{(}\PY{n}{N}\PY{p}{)}\PY{p}{:}
                  \PY{k}{if} \PY{n}{i} \PY{o}{!=} \PY{n}{x}\PY{p}{:}
                      \PY{k}{yield} \PY{n}{i}
          
          \PY{n}{first\PYZus{}NN} \PY{o}{=} \PY{n}{np}\PY{o}{.}\PY{n}{array}\PY{p}{(}\PY{p}{[}\PY{n}{np}\PY{o}{.}\PY{n}{sum}\PY{p}{(}\PY{p}{(}\PY{n}{descrs1}\PY{p}{[}\PY{n}{i}\PY{p}{]} \PY{o}{\PYZhy{}} \PY{n}{descrs2}\PY{p}{[}\PY{n}{matches}\PY{p}{[}\PY{n}{i}\PY{p}{,}\PY{l+m+mi}{1}\PY{p}{]}\PY{p}{]}\PY{p}{)}\PY{o}{*}\PY{o}{*}\PY{l+m+mi}{2}\PY{p}{)} \PY{k}{for} \PY{n}{i} \PY{o+ow}{in} \PY{n+nb}{range}\PY{p}{(}\PY{n}{N\PYZus{}frames1}\PY{p}{)}\PY{p}{]}\PY{p}{)}
          \PY{n}{distances2} \PY{o}{=} \PY{p}{[}\PY{p}{[}\PY{n}{np}\PY{o}{.}\PY{n}{sum}\PY{p}{(}\PY{p}{(}\PY{n}{descrs1}\PY{p}{[}\PY{n}{matches}\PY{p}{[}\PY{n}{i}\PY{p}{,}\PY{l+m+mi}{0}\PY{p}{]}\PY{p}{]} \PY{o}{\PYZhy{}} \PY{n}{descrs2}\PY{p}{[}\PY{n}{j}\PY{p}{]}\PY{p}{)}\PY{o}{*}\PY{o}{*}\PY{l+m+mi}{2}\PY{p}{)} \PY{k}{for} \PY{n}{j} \PY{o+ow}{in} \PY{n}{range2}\PY{p}{(}\PY{n}{N\PYZus{}frames2}\PY{p}{,} \PY{n}{matches}\PY{p}{[}\PY{n}{i}\PY{p}{,}\PY{l+m+mi}{1}\PY{p}{]}\PY{p}{)}\PY{p}{]} \PY{k}{for} \PY{n}{i} \PY{o+ow}{in} \PY{n+nb}{range}\PY{p}{(}\PY{n}{N\PYZus{}frames1}\PY{p}{)}\PY{p}{]}
          \PY{n}{second\PYZus{}NN} \PY{o}{=} \PY{n}{np}\PY{o}{.}\PY{n}{min}\PY{p}{(}\PY{n}{distances2}\PY{p}{,} \PY{n}{axis}\PY{o}{=}\PY{l+m+mi}{1}\PY{p}{)}
          
          \PY{n}{ratio} \PY{o}{=} \PY{n}{first\PYZus{}NN}\PY{o}{/}\PY{n}{second\PYZus{}NN}
\end{Verbatim}

    The ratio vector will be now used to retain only the matches that are
above a given threshold.

A value of \(NN_{threshold} = 0.8\) is often a good compromise between
losing too many matches and rejecting mismatches.

    \begin{Verbatim}[commandchars=\\\{\}]
{\color{incolor}In [{\color{incolor}108}]:} \PY{n}{NN\PYZus{}threshold}\PY{o}{=}\PY{l+m+mf}{0.8}
          
          \PY{n}{filtered\PYZus{}indices} \PY{o}{=} \PY{n}{np}\PY{o}{.}\PY{n}{flatnonzero}\PY{p}{(}\PY{n}{ratio}\PY{o}{\PYZlt{}}\PY{n}{NN\PYZus{}threshold}\PY{p}{)}
          \PY{n}{filtered\PYZus{}matches} \PY{o}{=} \PY{n}{matches}\PY{p}{[}\PY{n}{filtered\PYZus{}indices}\PY{p}{,}\PY{p}{:}\PY{p}{]}
\end{Verbatim}

    \begin{Verbatim}[commandchars=\\\{\}]
{\color{incolor}In [{\color{incolor}110}]:} \PY{c+c1}{\PYZsh{} plot matches}
          \PY{n}{plt}\PY{o}{.}\PY{n}{imshow}\PY{p}{(}\PY{n}{np}\PY{o}{.}\PY{n}{concatenate}\PY{p}{(}\PY{p}{(}\PY{n}{im1}\PY{p}{,} \PY{n}{im2}\PY{p}{)}\PY{p}{,} \PY{n}{axis}\PY{o}{=}\PY{l+m+mi}{1}\PY{p}{)}\PY{p}{)}
          \PY{k}{for} \PY{n}{idx} \PY{o+ow}{in} \PY{n+nb}{range}\PY{p}{(}\PY{n+nb}{len}\PY{p}{(}\PY{n}{filtered\PYZus{}matches}\PY{p}{)}\PY{p}{)}\PY{p}{:}
              \PY{n}{i}\PY{o}{=}\PY{n}{filtered\PYZus{}matches}\PY{p}{[}\PY{n}{idx}\PY{p}{,}\PY{l+m+mi}{0}\PY{p}{]}
              \PY{n}{j}\PY{o}{=}\PY{n}{filtered\PYZus{}matches}\PY{p}{[}\PY{n}{idx}\PY{p}{,}\PY{l+m+mi}{1}\PY{p}{]}
              \PY{c+c1}{\PYZsh{} plot dots at feature positions}
              \PY{n}{plt}\PY{o}{.}\PY{n}{gca}\PY{p}{(}\PY{p}{)}\PY{o}{.}\PY{n}{scatter}\PY{p}{(}\PY{p}{[}\PY{n}{frames1}\PY{p}{[}\PY{n}{i}\PY{p}{,}\PY{l+m+mi}{0}\PY{p}{]}\PY{p}{,}\PY{n}{im1}\PY{o}{.}\PY{n}{shape}\PY{p}{[}\PY{l+m+mi}{1}\PY{p}{]}\PY{o}{+}\PY{n}{frames2}\PY{p}{[}\PY{n}{j}\PY{p}{,}\PY{l+m+mi}{0}\PY{p}{]}\PY{p}{]}\PY{p}{,} \PY{p}{[}\PY{n}{frames1}\PY{p}{[}\PY{n}{i}\PY{p}{,}\PY{l+m+mi}{1}\PY{p}{]}\PY{p}{,}\PY{n}{frames2}\PY{p}{[}\PY{n}{j}\PY{p}{,}\PY{l+m+mi}{1}\PY{p}{]}\PY{p}{]}\PY{p}{,} \PY{n}{s}\PY{o}{=}\PY{l+m+mi}{5}\PY{p}{,} \PY{n}{c}\PY{o}{=}\PY{p}{[}\PY{l+m+mi}{0}\PY{p}{,}\PY{l+m+mi}{1}\PY{p}{,}\PY{l+m+mi}{0}\PY{p}{]}\PY{p}{)} 
              \PY{c+c1}{\PYZsh{} plot lines}
              \PY{n}{plt}\PY{o}{.}\PY{n}{plot}\PY{p}{(}\PY{p}{[}\PY{n}{frames1}\PY{p}{[}\PY{n}{i}\PY{p}{,}\PY{l+m+mi}{0}\PY{p}{]}\PY{p}{,}\PY{n}{im1}\PY{o}{.}\PY{n}{shape}\PY{p}{[}\PY{l+m+mi}{1}\PY{p}{]}\PY{o}{+}\PY{n}{frames2}\PY{p}{[}\PY{n}{j}\PY{p}{,}\PY{l+m+mi}{0}\PY{p}{]}\PY{p}{]}\PY{p}{,}\PY{p}{[}\PY{n}{frames1}\PY{p}{[}\PY{n}{i}\PY{p}{,}\PY{l+m+mi}{1}\PY{p}{]}\PY{p}{,}\PY{n}{frames2}\PY{p}{[}\PY{n}{j}\PY{p}{,}\PY{l+m+mi}{1}\PY{p}{]}\PY{p}{]}\PY{p}{,}\PY{n}{linewidth}\PY{o}{=}\PY{l+m+mf}{0.5}\PY{p}{)}
          \PY{n}{plt}\PY{o}{.}\PY{n}{show}\PY{p}{(}\PY{p}{)}
\end{Verbatim}

    \begin{center}
    \adjustimage{max size={0.9\linewidth}{0.9\paperheight}}{output_32_0.png}
    \end{center}
    { \hspace*{\fill} \\}
    
    \hypertarget{improving-sift-matching-ii-by-geometric-verification}{%
\subsection{Improving SIFT matching (ii) by geometric
verification}\label{improving-sift-matching-ii-by-geometric-verification}}

    In addition to the 2nd NN test, we can also require consistency between
the matches and a geometric transformation between the images. For the
moment we will look for matches that are consistent with a similarity
transformation:

\[\begin{bmatrix} x_2 \\ y_2 \end{bmatrix} = 
sR(\theta) \begin{bmatrix} x_1 \\ y_1 \end{bmatrix} + \begin{bmatrix} t_x \\ t_y \end{bmatrix} \]

which consists of a rotation by \(\theta\), an isotropic scaling
(i.e.~same in all directions) by s, and a translation by a vector
\((t_x, t_y)\). This transformation is specified by four parameters
\((s,\theta,t_x,t_y)\) and can be computed from a single correspondence
between SIFT detections in each image.

    \hypertarget{task-1.6}{%
\subsubsection{:: TASK 1.6 ::}\label{task-1.6}}

Given a detected feature with parameters \((x_1, y_1, s_1, \theta_1)\)
in image \(1\) matching a feature \((x_2, y_2, s_2, \theta_2)\) in image
\(2\), work out how to find out the parameters \((t_x,t_y,s,\theta)\) of
the transformation mapping points from image \(1\) to image \(2\).

    \[\theta=\theta_2-\theta_1\] \[s=\frac{s_2}{s_1}\]
\[\begin{bmatrix} t_x \\ t_y \end{bmatrix} = \begin{bmatrix} x_2-x_1. s_2/s_1 \\ y_2-y_1. s_2/s_1 \end{bmatrix}\]

    The matches consistent with a similarity can then be found using the
RANSAC algorithm, described by the following steps:

For each tentative correspondence in turn:

\begin{itemize}
\tightlist
\item
  compute the similarity transformation;
\item
  map all the SIFT detections in one image to the other using this
  transformation;
\item
  accept matches that are within a threshold distance to the mapped
  detection (inliers);
\item
  count the number of accepted matches;
\item
  optionally, fit a more accurate affine transformation or homography to
  the accepted matches and test re-validate the matches.
\end{itemize}

Finally, choose the transformation with the highest count of inliers

After this algorithm the inliers are consistent with the transformation
and are retained, and most mismatches should now be removed.

    \hypertarget{task-1.7}{%
\subsubsection{:: TASK 1.7 ::}\label{task-1.7}}

    \begin{Verbatim}[commandchars=\\\{\}]
{\color{incolor}In [{\color{incolor}111}]:} \PY{k}{def} \PY{n+nf}{ransac}\PY{p}{(}\PY{n}{frames1}\PY{p}{,}\PY{n}{frames2}\PY{p}{,}\PY{n}{matches}\PY{p}{,}\PY{n}{N\PYZus{}iters}\PY{o}{=}\PY{l+m+mi}{100}\PY{p}{,}\PY{n}{dist\PYZus{}thresh}\PY{o}{=}\PY{l+m+mi}{15}\PY{p}{)}\PY{p}{:}
            \PY{n}{max\PYZus{}inliers} \PY{o}{=} \PY{l+m+mi}{0}
            \PY{n}{tnf} \PY{o}{=} \PY{k+kc}{None}
              
            \PY{c+c1}{\PYZsh{} random sampling}
            \PY{k}{for} \PY{n}{it} \PY{o+ow}{in} \PY{n+nb}{range}\PY{p}{(}\PY{n}{N\PYZus{}iters}\PY{p}{)}\PY{p}{:}
                \PY{c+c1}{\PYZsh{} pick a random sample}
                \PY{n}{i} \PY{o}{=} \PY{n}{np}\PY{o}{.}\PY{n}{random}\PY{o}{.}\PY{n}{randint}\PY{p}{(}\PY{l+m+mi}{0}\PY{p}{,}\PY{n}{frames1}\PY{o}{.}\PY{n}{shape}\PY{p}{[}\PY{l+m+mi}{0}\PY{p}{]}\PY{p}{)}
                \PY{n}{x\PYZus{}1}\PY{p}{,} \PY{n}{y\PYZus{}1}\PY{p}{,} \PY{n}{s\PYZus{}1}\PY{p}{,} \PY{n}{theta\PYZus{}1} \PY{o}{=} \PY{n}{frames1}\PY{p}{[}\PY{n}{i}\PY{p}{,}\PY{p}{:}\PY{p}{]}
                \PY{n}{j} \PY{o}{=} \PY{n}{matches}\PY{p}{[}\PY{n}{i}\PY{p}{,}\PY{l+m+mi}{1}\PY{p}{]}
                \PY{n}{x\PYZus{}2}\PY{p}{,} \PY{n}{y\PYZus{}2}\PY{p}{,} \PY{n}{s\PYZus{}2}\PY{p}{,} \PY{n}{theta\PYZus{}2} \PY{o}{=} \PY{n}{frames2}\PY{p}{[}\PY{n}{j}\PY{p}{,}\PY{p}{:}\PY{p}{]}
                
                \PY{c+c1}{\PYZsh{} estimate transformation}
                \PY{n}{theta} \PY{o}{=} \PY{n}{theta\PYZus{}2} \PY{o}{\PYZhy{}} \PY{n}{theta\PYZus{}1}
                \PY{n}{s} \PY{o}{=} \PY{n}{s\PYZus{}2}\PY{o}{/}\PY{n}{s\PYZus{}1}
                \PY{n}{t\PYZus{}x} \PY{o}{=} \PY{n}{x\PYZus{}2}\PY{o}{\PYZhy{}}\PY{n}{x\PYZus{}1} \PY{o}{*} \PY{n}{s\PYZus{}2}\PY{o}{/}\PY{n}{s\PYZus{}1}
                \PY{n}{t\PYZus{}y} \PY{o}{=} \PY{n}{y\PYZus{}2}\PY{o}{\PYZhy{}}\PY{n}{y\PYZus{}1} \PY{o}{*} \PY{n}{s\PYZus{}2}\PY{o}{/}\PY{n}{s\PYZus{}1}
                
                \PY{c+c1}{\PYZsh{} evaluate estimated transformation}
                \PY{n}{X\PYZus{}1} \PY{o}{=} \PY{n}{frames1}\PY{p}{[}\PY{p}{:}\PY{p}{,}\PY{l+m+mi}{0}\PY{p}{]}
                \PY{n}{Y\PYZus{}1} \PY{o}{=} \PY{n}{frames1}\PY{p}{[}\PY{p}{:}\PY{p}{,}\PY{l+m+mi}{1}\PY{p}{]}
                \PY{n}{X\PYZus{}2} \PY{o}{=} \PY{n}{frames2}\PY{p}{[}\PY{n}{matches}\PY{p}{[}\PY{p}{:}\PY{p}{,}\PY{l+m+mi}{1}\PY{p}{]}\PY{p}{,}\PY{l+m+mi}{0}\PY{p}{]}
                \PY{n}{Y\PYZus{}2} \PY{o}{=} \PY{n}{frames2}\PY{p}{[}\PY{n}{matches}\PY{p}{[}\PY{p}{:}\PY{p}{,}\PY{l+m+mi}{1}\PY{p}{]}\PY{p}{,}\PY{l+m+mi}{1}\PY{p}{]}
                
                \PY{n}{X\PYZus{}1\PYZus{}prime} \PY{o}{=} \PY{n}{s}\PY{o}{*}\PY{p}{(}\PY{n}{X\PYZus{}1}\PY{o}{*}\PY{n}{np}\PY{o}{.}\PY{n}{cos}\PY{p}{(}\PY{n}{theta}\PY{p}{)}\PY{o}{\PYZhy{}}\PY{n}{Y\PYZus{}1}\PY{o}{*}\PY{n}{np}\PY{o}{.}\PY{n}{sin}\PY{p}{(}\PY{n}{theta}\PY{p}{)}\PY{p}{)}\PY{o}{+}\PY{n}{t\PYZus{}x}
                \PY{n}{Y\PYZus{}1\PYZus{}prime} \PY{o}{=} \PY{n}{s}\PY{o}{*}\PY{p}{(}\PY{n}{X\PYZus{}1}\PY{o}{*}\PY{n}{np}\PY{o}{.}\PY{n}{sin}\PY{p}{(}\PY{n}{theta}\PY{p}{)}\PY{o}{+}\PY{n}{Y\PYZus{}1}\PY{o}{*}\PY{n}{np}\PY{o}{.}\PY{n}{cos}\PY{p}{(}\PY{n}{theta}\PY{p}{)}\PY{p}{)}\PY{o}{+}\PY{n}{t\PYZus{}y}
                
                \PY{n}{dist} \PY{o}{=} \PY{n}{np}\PY{o}{.}\PY{n}{sqrt}\PY{p}{(}\PY{p}{(}\PY{n}{X\PYZus{}1\PYZus{}prime}\PY{o}{\PYZhy{}}\PY{n}{X\PYZus{}2}\PY{p}{)}\PY{o}{*}\PY{o}{*}\PY{l+m+mi}{2}\PY{o}{+}\PY{p}{(}\PY{n}{Y\PYZus{}1\PYZus{}prime}\PY{o}{\PYZhy{}}\PY{n}{Y\PYZus{}2}\PY{p}{)}\PY{o}{*}\PY{o}{*}\PY{l+m+mi}{2}\PY{p}{)}
                
                \PY{n}{inliers\PYZus{}indices} \PY{o}{=} \PY{n}{np}\PY{o}{.}\PY{n}{flatnonzero}\PY{p}{(}\PY{n}{dist}\PY{o}{\PYZlt{}}\PY{n}{dist\PYZus{}thresh}\PY{p}{)}
                \PY{n}{num\PYZus{}of\PYZus{}inliers} \PY{o}{=} \PY{n+nb}{len}\PY{p}{(}\PY{n}{inliers\PYZus{}indices}\PY{p}{)}
                
                \PY{c+c1}{\PYZsh{} keep if best}
                \PY{k}{if} \PY{n}{num\PYZus{}of\PYZus{}inliers}\PY{o}{\PYZgt{}}\PY{n}{max\PYZus{}inliers}\PY{p}{:}
                  \PY{n}{max\PYZus{}inliers}\PY{o}{=}\PY{n}{num\PYZus{}of\PYZus{}inliers}
                  \PY{n}{best\PYZus{}inliers\PYZus{}indices} \PY{o}{=} \PY{n}{inliers\PYZus{}indices}
                  \PY{n}{tnf} \PY{o}{=} \PY{p}{[}\PY{n}{t\PYZus{}x}\PY{p}{,}\PY{n}{t\PYZus{}y}\PY{p}{,}\PY{n}{s}\PY{p}{,}\PY{n}{theta}\PY{p}{]}
                  
            \PY{k}{return} \PY{p}{(}\PY{n}{tnf}\PY{p}{,}\PY{n}{best\PYZus{}inliers\PYZus{}indices}\PY{p}{)}
                
\end{Verbatim}

    \begin{Verbatim}[commandchars=\\\{\}]
{\color{incolor}In [{\color{incolor}112}]:} \PY{n}{tnf}\PY{p}{,}\PY{n}{inliers\PYZus{}indices}\PY{o}{=}\PY{n}{ransac}\PY{p}{(}\PY{n}{frames1}\PY{p}{,}\PY{n}{frames2}\PY{p}{,}\PY{n}{matches}\PY{p}{)}
          \PY{n}{filtered\PYZus{}matches} \PY{o}{=} \PY{n}{matches}\PY{p}{[}\PY{n}{inliers\PYZus{}indices}\PY{p}{,}\PY{p}{:}\PY{p}{]}
\end{Verbatim}

    \begin{Verbatim}[commandchars=\\\{\}]
{\color{incolor}In [{\color{incolor}113}]:} \PY{c+c1}{\PYZsh{} plot matches filtered with RANSAC}
          \PY{n}{plt}\PY{o}{.}\PY{n}{imshow}\PY{p}{(}\PY{n}{np}\PY{o}{.}\PY{n}{concatenate}\PY{p}{(}\PY{p}{(}\PY{n}{im1}\PY{p}{,}\PY{n}{im2}\PY{p}{)}\PY{p}{,}\PY{n}{axis}\PY{o}{=}\PY{l+m+mi}{1}\PY{p}{)}\PY{p}{)}
          \PY{k}{for} \PY{n}{idx} \PY{o+ow}{in} \PY{n+nb}{range}\PY{p}{(}\PY{n}{filtered\PYZus{}matches}\PY{o}{.}\PY{n}{shape}\PY{p}{[}\PY{l+m+mi}{0}\PY{p}{]}\PY{p}{)}\PY{p}{:}
              \PY{n}{i}\PY{o}{=}\PY{n}{filtered\PYZus{}matches}\PY{p}{[}\PY{n}{idx}\PY{p}{,}\PY{l+m+mi}{0}\PY{p}{]}
              \PY{n}{j}\PY{o}{=}\PY{n}{filtered\PYZus{}matches}\PY{p}{[}\PY{n}{idx}\PY{p}{,}\PY{l+m+mi}{1}\PY{p}{]}
              \PY{c+c1}{\PYZsh{} plot dots at feature positions}
              \PY{n}{plt}\PY{o}{.}\PY{n}{gca}\PY{p}{(}\PY{p}{)}\PY{o}{.}\PY{n}{scatter}\PY{p}{(}\PY{p}{[}\PY{n}{frames1}\PY{p}{[}\PY{n}{i}\PY{p}{,}\PY{l+m+mi}{0}\PY{p}{]}\PY{p}{,}\PY{n}{im1}\PY{o}{.}\PY{n}{shape}\PY{p}{[}\PY{l+m+mi}{1}\PY{p}{]}\PY{o}{+}\PY{n}{frames2}\PY{p}{[}\PY{n}{j}\PY{p}{,}\PY{l+m+mi}{0}\PY{p}{]}\PY{p}{]}\PY{p}{,} \PY{p}{[}\PY{n}{frames1}\PY{p}{[}\PY{n}{i}\PY{p}{,}\PY{l+m+mi}{1}\PY{p}{]}\PY{p}{,}\PY{n}{frames2}\PY{p}{[}\PY{n}{j}\PY{p}{,}\PY{l+m+mi}{1}\PY{p}{]}\PY{p}{]}\PY{p}{,} \PY{n}{s}\PY{o}{=}\PY{l+m+mi}{5}\PY{p}{,} \PY{n}{c}\PY{o}{=}\PY{p}{[}\PY{l+m+mi}{0}\PY{p}{,}\PY{l+m+mi}{1}\PY{p}{,}\PY{l+m+mi}{0}\PY{p}{]}\PY{p}{)} 
              \PY{c+c1}{\PYZsh{} plot lines}
              \PY{n}{plt}\PY{o}{.}\PY{n}{plot}\PY{p}{(}\PY{p}{[}\PY{n}{frames1}\PY{p}{[}\PY{n}{i}\PY{p}{,}\PY{l+m+mi}{0}\PY{p}{]}\PY{p}{,}\PY{n}{im1}\PY{o}{.}\PY{n}{shape}\PY{p}{[}\PY{l+m+mi}{1}\PY{p}{]}\PY{o}{+}\PY{n}{frames2}\PY{p}{[}\PY{n}{j}\PY{p}{,}\PY{l+m+mi}{0}\PY{p}{]}\PY{p}{]}\PY{p}{,}\PY{p}{[}\PY{n}{frames1}\PY{p}{[}\PY{n}{i}\PY{p}{,}\PY{l+m+mi}{1}\PY{p}{]}\PY{p}{,}\PY{n}{frames2}\PY{p}{[}\PY{n}{j}\PY{p}{,}\PY{l+m+mi}{1}\PY{p}{]}\PY{p}{]}\PY{p}{,}\PY{n}{linewidth}\PY{o}{=}\PY{l+m+mf}{0.5}\PY{p}{)}
          \PY{n}{plt}\PY{o}{.}\PY{n}{show}\PY{p}{(}\PY{p}{)}
\end{Verbatim}

    \begin{center}
    \adjustimage{max size={0.9\linewidth}{0.9\paperheight}}{output_41_0.png}
    \end{center}
    { \hspace*{\fill} \\}
    
    \hypertarget{part-2-compact-descriptors-for-image-retrieval}{%
\section{Part 2: Compact descriptors for image
retrieval}\label{part-2-compact-descriptors-for-image-retrieval}}

    In large scale retrieval the goal is to match a query image to a large
database of images (for example the WWW or Wikipedia).

The quality of an image match is measured as the number of geometrically
verified feature correspondences between the query and a database image.
While the techniques discussed in Part 1 are sufficient to do this, in
practice they require too much memory to store the SIFT descriptors for
all the detections in all the database images.

In this part we will see how we can compute a \emph{global} image
descriptor from the set of SIFT descriptors using the
bag-of-visual-words (BoVW) approach.

Then, we will see how these global descriptors can be used to rapidly
retrieve a shortlist of candidate database images given a query image.
Finally, we will see how to re-rank the shortlist of candidates using a
geometric verification technique that requires only the \emph{detector
frames} and their assigned visual word indices; remember the SIFT
descriptors are only used to compute the compact BoVW descriptors and
then discarded.

    \hypertarget{load-preprocessed-dataset-of-paintings}{%
\subsection{Load preprocessed dataset of
paintings}\label{load-preprocessed-dataset-of-paintings}}

    We will now load the preprocessed dataset of paintings. The construction
of this dataset has involved several steps.

\begin{enumerate}
\def\labelenumi{\arabic{enumi}.}
\tightlist
\item
  SIFT features were extracted from all paintings in the dataset
\item
  A global vocabulary of SIFT descriptors was computed using K-means
  clustering. These are the visual words of our dataset.
\item
  The SIFT features for each painting were assigned to the nearest word,
  and a compact descriptor was generated for each painting. This compact
  descriptor consists in the normalized histogram of words. The
  histogram normalization itself involves 3 different steps:

  \begin{itemize}
  \item
    \begin{enumerate}
    \def\labelenumii{\alph{enumii})}
    \tightlist
    \item
      TF-IDF weighting: each word is re-weighted according to its TF-IDF
      value. This removes weights to words that are very common and
      therefore not too descriptive.
    \end{enumerate}
  \item
    \begin{enumerate}
    \def\labelenumii{\alph{enumii})}
    \setcounter{enumii}{1}
    \tightlist
    \item
      Square-rooting: each element is square-rooted
    \end{enumerate}
  \item
    \begin{enumerate}
    \def\labelenumii{\alph{enumii})}
    \setcounter{enumii}{2}
    \tightlist
    \item
      L2-normalization: The whole histogram is L2-normalized.
    \end{enumerate}
  \end{itemize}
\end{enumerate}

    \begin{Verbatim}[commandchars=\\\{\}]
{\color{incolor}In [{\color{incolor}2}]:} \PY{n}{imdb}\PY{o}{=}\PY{n}{loadmat}\PY{p}{(}\PY{l+s+s1}{\PYZsq{}}\PY{l+s+s1}{paintings\PYZus{}imdb\PYZus{}SIFT\PYZus{}10k\PYZus{}preprocessed.mat}\PY{l+s+s1}{\PYZsq{}}\PY{p}{)}
        
        \PY{n}{feature\PYZus{}vocab}\PY{o}{=}\PY{n}{np}\PY{o}{.}\PY{n}{transpose}\PY{p}{(}\PY{n}{imdb}\PY{p}{[}\PY{l+s+s1}{\PYZsq{}}\PY{l+s+s1}{vocab}\PY{l+s+s1}{\PYZsq{}}\PY{p}{]}\PY{p}{)}
        \PY{n}{imdb\PYZus{}hists}\PY{o}{=}\PY{n}{imdb}\PY{p}{[}\PY{l+s+s1}{\PYZsq{}}\PY{l+s+s1}{index}\PY{l+s+s1}{\PYZsq{}}\PY{p}{]}
        \PY{n}{imdb\PYZus{}tfidf}\PY{o}{=}\PY{n}{imdb}\PY{p}{[}\PY{l+s+s1}{\PYZsq{}}\PY{l+s+s1}{idf}\PY{l+s+s1}{\PYZsq{}}\PY{p}{]}
        
        \PY{n}{imdb\PYZus{}url} \PY{o}{=} \PY{k}{lambda} \PY{n}{idx}\PY{p}{:} \PY{n}{imdb}\PY{p}{[}\PY{l+s+s1}{\PYZsq{}}\PY{l+s+s1}{images}\PY{l+s+s1}{\PYZsq{}}\PY{p}{]}\PY{p}{[}\PY{l+m+mi}{0}\PY{p}{]}\PY{p}{[}\PY{l+m+mi}{0}\PY{p}{]}\PY{p}{[}\PY{l+m+mi}{3}\PY{p}{]}\PY{p}{[}\PY{l+m+mi}{0}\PY{p}{]}\PY{p}{[}\PY{n}{idx}\PY{p}{]}\PY{o}{.}\PY{n}{item}\PY{p}{(}\PY{p}{)}
        
        \PY{n}{num\PYZus{}words} \PY{o}{=} \PY{n}{feature\PYZus{}vocab}\PY{o}{.}\PY{n}{shape}\PY{p}{[}\PY{l+m+mi}{0}\PY{p}{]}
        \PY{n}{num\PYZus{}paintings} \PY{o}{=} \PY{n}{imdb\PYZus{}hists}\PY{o}{.}\PY{n}{shape}\PY{p}{[}\PY{l+m+mi}{1}\PY{p}{]}
        
        \PY{n+nb}{print}\PY{p}{(}\PY{l+s+s1}{\PYZsq{}}\PY{l+s+s1}{The vocabulary of SIFT features contains }\PY{l+s+si}{\PYZpc{}s}\PY{l+s+s1}{  visual words}\PY{l+s+s1}{\PYZsq{}} \PY{o}{\PYZpc{}} \PY{n}{num\PYZus{}words}\PY{p}{)}
        \PY{n+nb}{print}\PY{p}{(}\PY{l+s+s1}{\PYZsq{}}\PY{l+s+s1}{The dictionary index contains }\PY{l+s+si}{\PYZpc{}s}\PY{l+s+s1}{ histograms corresponding to each painting in the dataset}\PY{l+s+s1}{\PYZsq{}} \PY{o}{\PYZpc{}} \PY{n}{num\PYZus{}paintings}\PY{p}{)}
        \PY{n+nb}{print}\PY{p}{(}\PY{l+s+s1}{\PYZsq{}}\PY{l+s+s1}{The tdf\PYZhy{}idf vector has shape }\PY{l+s+s1}{\PYZsq{}}\PY{o}{+}\PY{n+nb}{str}\PY{p}{(}\PY{n}{imdb\PYZus{}tfidf}\PY{o}{.}\PY{n}{shape}\PY{p}{)}\PY{p}{)}
\end{Verbatim}

    \begin{Verbatim}[commandchars=\\\{\}]
The vocabulary of SIFT features contains 10000  visual words
The dictionary index contains 1703 histograms corresponding to each painting in the dataset
The tdf-idf vector has shape (10000, 1)

    \end{Verbatim}

    \begin{Verbatim}[commandchars=\\\{\}]
{\color{incolor}In [{\color{incolor}3}]:} \PY{n}{painting} \PY{o}{=} \PY{n}{imread}\PY{p}{(}\PY{l+s+s1}{\PYZsq{}}\PY{l+s+s1}{data/queries/mistery\PYZhy{}painting1.jpg}\PY{l+s+s1}{\PYZsq{}}\PY{p}{)}
        
        \PY{p}{[}\PY{n}{frames}\PY{p}{,} \PY{n}{descrs}\PY{p}{]} \PY{o}{=} \PY{n}{cyvlfeat}\PY{o}{.}\PY{n}{sift}\PY{o}{.}\PY{n}{sift}\PY{p}{(}\PY{n}{rgb2gray}\PY{p}{(}\PY{n}{painting}\PY{p}{)}\PY{p}{,} \PY{n}{peak\PYZus{}thresh}\PY{o}{=}\PY{l+m+mf}{0.01}\PY{p}{)}
        
        \PY{n}{plt}\PY{o}{.}\PY{n}{imshow}\PY{p}{(}\PY{n}{painting}\PY{p}{)}
        \PY{n}{plotframes}\PY{p}{(}\PY{n}{frames}\PY{p}{,}\PY{n}{linewidth}\PY{o}{=}\PY{l+m+mi}{1}\PY{p}{)}
\end{Verbatim}

    \begin{center}
    \adjustimage{max size={0.9\linewidth}{0.9\paperheight}}{output_47_0.png}
    \end{center}
    { \hspace*{\fill} \\}
    
    \hypertarget{task-2.1}{%
\subsubsection{:: TASK 2.1 ::}\label{task-2.1}}

Construct a KDTree of the vocabulary for fast NN search. Then use the
KDTree to find the closest word in the vocabulary to each descriptor of
the query image.

    \begin{Verbatim}[commandchars=\\\{\}]
{\color{incolor}In [{\color{incolor}4}]:} \PY{k+kn}{from} \PY{n+nn}{sklearn}\PY{n+nn}{.}\PY{n+nn}{neighbors} \PY{k}{import} \PY{n}{KDTree}
        
        \PY{n}{tree} \PY{o}{=} \PY{n}{KDTree}\PY{p}{(}\PY{n}{feature\PYZus{}vocab}\PY{p}{)}              
        \PY{n}{dist}\PY{p}{,} \PY{n}{ind} \PY{o}{=} \PY{n}{tree}\PY{o}{.}\PY{n}{query}\PY{p}{(}\PY{n}{descrs}\PY{p}{,} \PY{n}{k}\PY{o}{=}\PY{l+m+mi}{2}\PY{p}{)}                
        \PY{n}{closest\PYZus{}words} \PY{o}{=} \PY{n}{feature\PYZus{}vocab}\PY{p}{[}\PY{n}{ind}\PY{p}{[}\PY{p}{:}\PY{p}{,}\PY{l+m+mi}{0}\PY{p}{]}\PY{p}{]}
        
        \PY{n}{plt}\PY{o}{.}\PY{n}{plot}\PY{p}{(}\PY{n}{ind}\PY{p}{[}\PY{p}{:}\PY{p}{,}\PY{l+m+mi}{0}\PY{p}{]}\PY{p}{,} \PY{l+s+s1}{\PYZsq{}}\PY{l+s+s1}{+}\PY{l+s+s1}{\PYZsq{}}\PY{p}{)}
        \PY{n}{plt}\PY{o}{.}\PY{n}{xlabel}\PY{p}{(}\PY{l+s+s2}{\PYZdq{}}\PY{l+s+s2}{ID word in painting}\PY{l+s+s2}{\PYZdq{}}\PY{p}{)}
        \PY{n}{plt}\PY{o}{.}\PY{n}{ylabel}\PY{p}{(}\PY{l+s+s2}{\PYZdq{}}\PY{l+s+s2}{ID word in vocabulary}\PY{l+s+s2}{\PYZdq{}}\PY{p}{)}
\end{Verbatim}

            \begin{Verbatim}[commandchars=\\\{\}]
{\color{outcolor}Out[{\color{outcolor}4}]:} Text(0,0.5,'ID word in vocabulary')
\end{Verbatim}
        
    \begin{center}
    \adjustimage{max size={0.9\linewidth}{0.9\paperheight}}{output_49_1.png}
    \end{center}
    { \hspace*{\fill} \\}
    
    \hypertarget{task-2.2}{%
\subsubsection{:: TASK 2.2 ::}\label{task-2.2}}

Compute the compact BoVW descriptor of the query image.

    \begin{Verbatim}[commandchars=\\\{\}]
{\color{incolor}In [{\color{incolor}30}]:} \PY{n}{query\PYZus{}hist}\PY{o}{=}\PY{n}{np}\PY{o}{.}\PY{n}{zeros}\PY{p}{(}\PY{n}{num\PYZus{}words}\PY{p}{)}
         
         \PY{k}{for} \PY{n}{i} \PY{o+ow}{in} \PY{n}{ind}\PY{p}{[}\PY{p}{:}\PY{p}{,}\PY{l+m+mi}{0}\PY{p}{]}\PY{p}{:}
             \PY{n}{query\PYZus{}hist}\PY{p}{[}\PY{n}{i}\PY{p}{]} \PY{o}{+}\PY{o}{=} \PY{l+m+mi}{1}
\end{Verbatim}

    \begin{Verbatim}[commandchars=\\\{\}]
{\color{incolor}In [{\color{incolor}31}]:} \PY{c+c1}{\PYZsh{} process histogram}
         \PY{n}{query\PYZus{}hist} \PY{o}{=} \PY{n}{np}\PY{o}{.}\PY{n}{multiply}\PY{p}{(}\PY{n}{query\PYZus{}hist}\PY{p}{,} \PY{n}{imdb\PYZus{}tfidf}\PY{p}{[}\PY{p}{:}\PY{p}{,}\PY{l+m+mi}{0}\PY{p}{]}\PY{p}{)}
         \PY{n}{query\PYZus{}hist} \PY{o}{=} \PY{n}{np}\PY{o}{.}\PY{n}{sqrt}\PY{p}{(}\PY{n}{query\PYZus{}hist}\PY{p}{)}
         \PY{n}{query\PYZus{}hist} \PY{o}{=} \PY{n}{query\PYZus{}hist}\PY{o}{/}\PY{n}{np}\PY{o}{.}\PY{n}{linalg}\PY{o}{.}\PY{n}{norm}\PY{p}{(}\PY{n}{query\PYZus{}hist}\PY{p}{)}
\end{Verbatim}

    \hypertarget{task-2.3}{%
\subsubsection{:: TASK 2.3 ::}\label{task-2.3}}

Compute the matching score with each image from the database.

    \begin{Verbatim}[commandchars=\\\{\}]
{\color{incolor}In [{\color{incolor}75}]:} \PY{n}{scores} \PY{o}{=} \PY{n}{np}\PY{o}{.}\PY{n}{zeros}\PY{p}{(}\PY{n}{num\PYZus{}paintings}\PY{p}{)}
         \PY{k}{for} \PY{n}{i} \PY{o+ow}{in} \PY{n+nb}{range}\PY{p}{(}\PY{n}{num\PYZus{}paintings}\PY{p}{)}\PY{p}{:}
             \PY{n}{scores}\PY{p}{[}\PY{n}{i}\PY{p}{]} \PY{o}{=} \PY{n}{np}\PY{o}{.}\PY{n}{sum}\PY{p}{(}\PY{n+nb}{abs}\PY{p}{(}\PY{n}{query\PYZus{}hist} \PY{o}{\PYZhy{}} \PY{n}{np}\PY{o}{.}\PY{n}{ravel}\PY{p}{(}\PY{n}{imdb\PYZus{}hists}\PY{p}{[}\PY{p}{:}\PY{p}{,}\PY{n}{i}\PY{p}{]}\PY{o}{.}\PY{n}{todense}\PY{p}{(}\PY{p}{)}\PY{p}{)}\PY{p}{)}\PY{p}{)}
\end{Verbatim}

    \begin{Verbatim}[commandchars=\\\{\}]
{\color{incolor}In [{\color{incolor}73}]:} \PY{c+c1}{\PYZsh{} sort in descending order}
         \PY{n}{scores\PYZus{}sorted\PYZus{}idx} \PY{o}{=} \PY{n}{np}\PY{o}{.}\PY{n}{argsort}\PY{p}{(}\PY{o}{\PYZhy{}}\PY{n}{scores}\PY{p}{)}
         \PY{n}{scores\PYZus{}sorted} \PY{o}{=} \PY{n}{scores}\PY{o}{.}\PY{n}{ravel}\PY{p}{(}\PY{p}{)}\PY{p}{[}\PY{n}{scores\PYZus{}sorted\PYZus{}idx}\PY{p}{]}
\end{Verbatim}

    \begin{Verbatim}[commandchars=\\\{\}]
{\color{incolor}In [{\color{incolor}77}]:} \PY{c+c1}{\PYZsh{} plot top matches}
         \PY{n}{N}\PY{o}{=}\PY{l+m+mi}{10}
         \PY{n}{top\PYZus{}N\PYZus{}idx} \PY{o}{=} \PY{n}{scores\PYZus{}sorted\PYZus{}idx}\PY{o}{.}\PY{n}{ravel}\PY{p}{(}\PY{p}{)}\PY{p}{[}\PY{p}{:}\PY{n}{N}\PY{p}{]}
         \PY{n}{plt}\PY{o}{.}\PY{n}{figure}\PY{p}{(}\PY{n}{figsize}\PY{o}{=}\PY{p}{(}\PY{l+m+mi}{20}\PY{p}{,}\PY{l+m+mi}{15}\PY{p}{)}\PY{p}{)}
         \PY{k}{for} \PY{n}{i} \PY{o+ow}{in} \PY{n+nb}{range}\PY{p}{(}\PY{n}{N}\PY{p}{)}\PY{p}{:}
           \PY{c+c1}{\PYZsh{} download images}
           \PY{n}{url} \PY{o}{=} \PY{n}{imdb\PYZus{}url}\PY{p}{(}\PY{n}{top\PYZus{}N\PYZus{}idx}\PY{p}{[}\PY{n}{i}\PY{p}{]}\PY{p}{)}      
           \PY{k}{with} \PY{n}{urlopen}\PY{p}{(}\PY{n}{url}\PY{p}{)} \PY{k}{as} \PY{n}{file}\PY{p}{:}
               \PY{n}{img} \PY{o}{=} \PY{n}{imread}\PY{p}{(}\PY{n}{file}\PY{p}{,} \PY{n}{mode}\PY{o}{=}\PY{l+s+s1}{\PYZsq{}}\PY{l+s+s1}{RGB}\PY{l+s+s1}{\PYZsq{}}\PY{p}{)}
           \PY{c+c1}{\PYZsh{} choose subplot}
           \PY{n}{plt}\PY{o}{.}\PY{n}{subplot}\PY{p}{(}\PY{n+nb}{int}\PY{p}{(}\PY{n}{np}\PY{o}{.}\PY{n}{ceil}\PY{p}{(}\PY{n}{N}\PY{o}{/}\PY{l+m+mi}{5}\PY{p}{)}\PY{p}{)}\PY{p}{,}\PY{l+m+mi}{5}\PY{p}{,}\PY{n}{i}\PY{o}{+}\PY{l+m+mi}{1}\PY{p}{)}
           \PY{c+c1}{\PYZsh{} plot}
           \PY{n}{plt}\PY{o}{.}\PY{n}{imshow}\PY{p}{(}\PY{n}{img}\PY{p}{)}
           \PY{n}{plt}\PY{o}{.}\PY{n}{axis}\PY{p}{(}\PY{l+s+s1}{\PYZsq{}}\PY{l+s+s1}{off}\PY{l+s+s1}{\PYZsq{}}\PY{p}{)}
           \PY{n}{plt}\PY{o}{.}\PY{n}{title}\PY{p}{(}\PY{l+s+s1}{\PYZsq{}}\PY{l+s+s1}{score }\PY{l+s+si}{\PYZpc{}1.2f}\PY{l+s+s1}{\PYZsq{}} \PY{o}{\PYZpc{}} \PY{n}{scores\PYZus{}sorted}\PY{o}{.}\PY{n}{ravel}\PY{p}{(}\PY{p}{)}\PY{p}{[}\PY{n}{i}\PY{p}{]}\PY{p}{)}
           
\end{Verbatim}

    \begin{center}
    \adjustimage{max size={0.9\linewidth}{0.9\paperheight}}{output_56_0.png}
    \end{center}
    { \hspace*{\fill} \\}
    
    \hypertarget{authorship-statement}{%
\subsection{AUTHORSHIP STATEMENT}\label{authorship-statement}}

I declare that the preceding work was the sole result of my own effort
and that I have not used any code or results from third-parties.

Pierre-Louis Guhur


    % Add a bibliography block to the postdoc
    
    
    
    \end{document}
