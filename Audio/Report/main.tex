\documentclass{article}
\usepackage[utf8]{inputenc}
\usepackage{mathtools}
\title{Neural Network Approach in Music Source Separation}
\author{Pierre-Louis Guhur }
\date{February 2019}

\begin{document}

\maketitle

\begin{abstract}
    Given a mixture of signals, source separation aims at finding each of the original signal.
    Music is one of the important application field of source separation, as each instrument has a specific signature.
    However, the challenge remains open. Neural networks seem promising to classify short term audio samples among a panel of given instruments.
    A deep analysis of the audio properties provide a clearer understanding of required pre-processing steps.
\end{abstract}

\section{Introduction}

Source separation is a well-known research field in audio processing. It aims at recovering original signals on a mixture made of them.
In addition to the difficulty of the task itself, research in source separation is also motivated for its applications in automatic
speech recognition, and fundamental frequency estimation.
This research field has different setup. For example, in the case of blind signal separation,~\cite{comon2010handbook} the algorithm does not know the number of sources. 

In this report, we focus on the case of music source separation.
Observing a mixture of several audio signals $s_j$ at the discrete time index $n$: 
$$x(n) = \sum_{j\in \mathcal{J}} {s_j(n)},$$ 
the mission consists of finding estimates $\hat{s}_j$ of $s_j$. 
Recent papers investigated the use of deep neural networks (DNN), in order to regress amplitude features of each estimated signals $\hat{s}_j$ using  short-therm audio samples.

% #TODO add figure like Fig.1 

The originality of Muth et al.'s approach \cite{muth2018improving} is to successfully employ the phase as in an input feature in the prediction of the amplitude of each estimated signal $\hat{s}_j$.
In particular, they show that an adequate pre-processing step allows them to enhance state-of-the-art results on the DSD100 dataset.

The DSD100 dataset is composed of 100 songs split half between the training set and the testing set. For each song, the mixture signal and the original source signals (vocals, drums, bass, and other) are given.
This dataset is widely used in the literature, since the creation of a SiSEC challenge. 

The first Section of this report dedicates itself to the use a DNN-based approach for separating musical sources. Then, experiments are presented, which lead to an analysis of the approach's limit.  
Finally, enhancements inspired by recent papers are proposed. 




\section{Method}
\label{sec:background}

%\begin{center}
%  \smartdiagramset{border color=none, 
%    uniform color list=teal!60 for 4 items,
%  }
%  \smartdiagram[flow diagram:horizontal]{}
%\end{center}

%\begin{tikzpicture}[node distance = 2cm, auto]
%  % Place nodes
%  \node  (init) {$x[n]$};
%  \node [block, right of=init](dnnv) {DNN vocals} ;
%  \node [block, below of=dnnv](dnnb) {DNN bass} ;
%  \node [block, below of=dnnb](dnnd) {DNN drums} ;
%  \node [block, below of=dnnd](dnno) {DNN other} ;
%  \node [right of=dnnv](outv) {$\hat{s}_{vocals}[n]$} ;
%  \node [right of=dnnb](outb) {$\hat{s}_{bass}[n]$} ;
%  \node [right of=dnnd](outd) {$\hat{s}_{drums}[n]$} ;
%  \node [right of=dnno](outo) {$\hat{s}_{other}[n]$} ;
%
%  % Draw edges
%  \path [line, bend left] (init) -- (dnnv);
%  \path [line, bend left] (init) -- (dnnb);
%  \path [line] (init) -- (dnnd);
%  \path [line] (init) -- (dnno);
%  \path [line] (dnnv) -- (outv);
%  \path [line] (dnnb) -- (outb);
%  \path [line] (dnnd) -- (outd);
%  \path [line] (dnno) -- (outo);
%\end{tikzpicture}

\begin{figure}
  \centering
  \includegraphics[width=0.5\columnwidth]{mss-basic.png}
  \label{fig:mss-basic}
  \caption{DNN could predict music sources sample per sample}
\end{figure}

In this Section, we present how DNNs are able to solve the music source separation challenge, as in the Figure~\ref{fig:mss-basic}. One naive approach is to use a neural network to predict the estimated source $\hat{s}$ given a sample of the mixture signal $x[n]$. It is noticeable that this simple structure is not using any relation between each samples from $x$. 

Unfortunately, training such neural networks appear particularly difficult. 
Even for human beings, it is impossible to distinguish any pattern on only one sample without any context.
To bring more context to the neural network, the DNN could be fed with a sequence of samples $(x[n+i])_{i\in [-N, N]}$.
Music patterns, such as fading, last of roughly 5 seconds. With a sample rate of 44.1 kHz, the sequence would have a length of $441k$ samples. 
At that size, the curse of dimensionalityi~\cite{mallat} makes it difficult to learn patterns.

%The neural network can not find the invariances between each instruments.

\subsection{Short-term Fourier transform}

Instead, it seems preferable to help the neural network to figure out some invariances using pre-processing steps. 
A widely employed in audio consists in transforming an audio signal into the short-term Fourier transform (STFT). The STFT is a collection of Fourier transform on short section of a signal (typically of 20 ms). The STFT is particularly employed in audio processing, as the signal decomposition is varying around the time axis. 
The STFT is directly derived from the Fourier transform by employing a window function and a delay function:

$$\mathbf{STFT}{x(t)}(\tau,\omega) = X(\tau,\omega) = e^{j\omega t/2}\int_{-\infty}^\infty {x(t)w(t-\tau)e^(-jwt)dt}.$$

Thereafter, we use a Gaussian window function:

$$h(t) = \lambda^{-1/2}\pi^{-1/4}e^{-t^2/(2\lambda^2)}.$$

The STFT of a signal is then a two dimension and complex signal:
$$X(\omega, \tau) = A(\omega, \tau) e^{j\phi(\omega, \tau)},$$
with $\phi$ and $A$ the phase and the amplitude.


\begin{figure}
  \centering
  \includegraphics[width=0.9\columnwidth]{mss-amp.png}
  \label{fig:mss-amp}
  \caption{Samples are pre-processed with STFT to ease the appearance of invariances}
\end{figure}

Using an STFT, one solution is to fed the DNN with the amplitude and the phase of the mixture signal, as illustrated in Figure~\ref{fig:mss-amp}.
In this case, the DNN predict the amplitude and the phase of each source. Then the inverse STFT (iSTFT) is applied on the predicted amplitude and phase to reconstruct the source. 


\subsection{Sharing the phase of the mixture signal}

\begin{figure}
  \centering
  \includegraphis[width=0.9\columnwidth]{approx-phase.png}
  \label{fig:approx-phase.png}
  \caption{Comparison between the voice source, the signal reconstructed after an STFT transformation, the signal reconstructed with the phase of the mixture, and with the phase of the drums source}

Predicting the phase is difficult because of its intrinsic discontinuous shape. 
In the context of music source separation, it is generally assumed~\cite{chandna2017monoaural}, \cite{jansson2017singing} that the phase of the mixture signal is the same than each source's phase.
This approximated is illustrated in Figure~\cite{fig:approx-phase}. 

One could wander what is the impact of such transformations on the signal. By computing the signal-to-distortion ratio (SDR), defined as the ratio between the reconstructed signal's power and the original source's power, I observed a loss of 


\subsection{Relations between phase and amplitude} 

The assumption of a Gaussian window allows us to draw a relation between the phase and the amplitude thanks to the work of \cite{auger2012phase}:

\begin{equation}
    [
    \left\{ 
        \begin{array}{ll}
            \frac{\partial }{\partial t}\phi(\omega, \tau) &= \lambda^{-2}   \frac{\partial}{\partial \omega} \log(A(\omega, \tau)) + \frac{\omega}{2} \\
            \frac{\partial d}{\partial \omega}\phi(\omega, \tau) &= -\lambda^{-2}   \frac{\partial}{\partial t} \log(A(\omega, \tau)) - \frac{t}{2}
        \end{array}
    \right.
    ]
\end{equation}

Actually, similar expressions exist for higher order derivatives and for any window function having a certain regularity. 
This relationship provides the intuition to the authors that the derivative of the phase might help a neural network to predict the amplitude.
At the same time, neural networks have stronger difficulties to learn a model when the inputs are correlated.
That is why, 


\subsection{In the discrete time scale}

Numerical audio samples are discretized for discrete time steps $n$, and in this case:
$$\mathbf {STFT} \{x[n]\}(m,\omega ) = X(m,\omega )=e^{j\omega t/2}\sum _{n=-\infty }^{\infty }x[n]w[n-m]e^{-j\omega n}.$$

The intuition of using features extracted from derivatives of the phase required to approximate the discrete time derivatives using finite differences. Thereafter, we use:

$$\Delta_\tau \phi(k, m) = \phi(k, m) - \phi(k,m-1)$$
$$\Delta_\omega \phi(k, m) = \phi(k, m) - \phi(k-1,m)$$



\subsection{Pre-processing}

Pre-processing is used in neural networks to help an optimization algorithm to find symmetries in the training set. Pre-processing is all the more important when the dataset is small. 

In the article of Muth et al., the processing consists in computing the finite-difference derivative and then fixes shits on their distribution.

The authors observed that the distributions of $\Delta_\tau \phi$ and $\Delta_\omega \phi$ are illustrated in Figure~\ref{fig:dist-phi}. We observe that $\Delta_\omega \phi$ has a systematic shift of $\pi$, whereas $\Delta_\tau \phi$ has a shift of $2\pi k\frac{n_0}{N}$, where $n_0$ is the hop size, namely the number of taps between two samples $k$ and $k+1$.

The shift of $\Delta_\tau \phi$ is explained by the shift theorem of the discrete Fourier transform \cite{smith2007mathematics}:
$$x(n-n_0) \xrightarrow{\text{DFT}}
e^{j\frac{2\pi}{N}kn_0}X(k).$$

\subsection{Predicting the amplitude with a neural network}


\section{Experiments}
\label{sec:xp}

This Section presents some experiments obtained with the previously described method.

\subsection{Training the neural networks}

The DNN proposed in the Figure~\ref{fig:nn} was trained
on the DSD100 dataset.
First at all, STFT features were computed. With an hop size of $1024$ samples,
around 180k samples were computed. Samples were then grouped into batches of 256 samples.
After each batch, the model was optimized according to the mean square error.
The Figure~\ref{fig:train-basic} shows the accuracy of the model over a few epochs.
Accuracy is compared with the validation set to detect whether the model is
over-fitting or not.

\begin{figure}
  \centering
  \includegraphics[width=0.8\columnwidth]{train-basic.png}
  \label{fig:train-basic}
  \caption{Loss during the training}
\end{figure}

\subsection{Improving the model}

It appears that the network was completely overfitting. Overfitting might be induced
by a few factors: the high number of parameters, the correlation between the input
features, the heavy pre-processing steps.

\begin{figure}
  \centering
  \includegraphics[width=0.9\columnwidth]{final-architecture.png}
  \label{fig:final-architecture}
  \caption{Improvements of the neural network}
\end{figure}

Consequently, I brought some changes to the neural network, as shown in Figure
\ref{fig:final-architecture}. The dense layers and the activation functions are
replaced by blocks containing a dense layer, a batch normalization layer,
 a dropout layer and an activation function.
The dense layers are using multi-channels, such that
the number of parameters were narrowed down to around 550k.
The Figure~\ref{fig:train-final} shows the training loss of the new model.
A significant improvement was obtained.

\begin{figure}
  \centering
  \includegraphics[width=0.6\columnwidth]{train-final.png}
  \label{fig:train-final}
  \caption{Loss during the training after the improvement of the network}
\end{figure}

\subsection{Signal to distortion ratio}

Literature is commonly using 3 metrics, SDR (signal to distortion ratio),
SIR (signal to interferences ratio) and SAR (signal to artifacts ratio).
Errors in source separation might come from two origins:
noise due to mis-separation, called interferences, and noise to the reconstruction
algorithm itself, called crosstalk.
%Typically, increasing the STFT window size tends to increase SIR, but decrease SAR.
The Table~\ref{tab:res} presents the metrics SDR for the case with the original
neural network, the final architecture with and without the correction
of the feature distributions.
The methods "LSTM" and "WaveNet" are presented in the Section~\ref{sec:related}.

\begin{table}
  \centering
  \begin{tabular}{|c| c c c c| }
  \hline
    Sources & Vocal & Bass & Drums & Other \\
  \hline
    Original & 2.32 & 1.01 & 1.39 & 1.32 \\
  \hline
    No rectification & 3.48 & 2.15 & 2.71 & 2.O1 \\
  \hline
    Rectified distributions & 3.51 & 2.10 & 2.72 & 2.12 \\
  \hline
    WaveNet & 4.60 & 2.49 & 4.60 & 0.54 \\
  \hline
    LSTM & 6.31 & 3.73 & 5.46 & 4.33 \\
  \hline
  \end{tabular}
  \label{tab:res}
    \caption{Signal to distortion ratio for the method proposed in the report compared with state-of-the-art methods}
\end{table}


\subsection{Analysis of a result}

\begin{figure}
  \centering
  \includegraphics[width=0.8\columnwidth]{bass-src.png}
  \label{fig:bass-src}
  \caption{Source signal of the bass}

  \includegraphics[width=0.8\columnwidth]{bass-pred.png}
  \label{fig:bass-pred}
  \caption{Source signal of the prediction}
\end{figure}

For one example song, the Figure~\ref{fig:bass-src} presents the bass source,
whereas the Figure~\ref{fig:bass-pred} shows the prediction source.
We can observe that long patterns such as fading at the beginning are not learned.
One explanation is that the time context is too short for these patterns.
Furthermore, the neural network tends to empty gaps, creating a feeling of
a white noise.

I tried to compute the source "Other" as the difference between the signal of
the mixture and the other source signals, but it provides a very distorted signal.
This shows that errors in other source signals remain significant.

\section{Related work}
\label{sec:related}
% https://github.com/ybayle/awesome-deep-learning-music/blob/master/README.md

While the ideas of using MLP and RNN for music-based applications were already proposed  by Todd \cite{Todd1988} and Lewis \cite{Lewis1988} in 1988,  music source separation became particularly popular from 2014 thanks to the ``Singing Voice Separation" challenge proposed by MIREX.



Chandna et al. \cite{chandna2017monoaural} are using a convolutional neural network between the time $\tau$ and the frequency $f$.

Time frequency masking



Let the neural network discovers more invariances in data structures thanks to data augmentation/data generation.


\section{Conclusion}

Since the emergence of deep learning algorithms, music source separation has gained
a strong interest in the literature.
Most of the algorithms are using an STFT representation of the mixture signal,
instead of its waveform.
Predicting the phase is difficult due to discontinuous shape.
However, the phase of the mixture is a correct approximation of each source's phase.
State-of-the-art methods focus then in predicting the magnitude of each source.

In this perspective, the contribution of Muth et al. consists in using phase features
as an input to the DNNs. The phase can not be used directly, but its frequency
and temporal derivatives can help to recover the magnitude of each source signal.
This affirmation relies on an analytical relation between these derivatives and
the magnitude.

In this report, we investigated their contribution, showing how to correct the
distributions of the phase features, but also how to improve the DNN to limit
overfitting. Experiments provided satisfying results, but remained incomplete
for long-term patterns such as fading.


{\small
\bibliographystyle{abbrv}
\bibliography{biblio}
}

\end{document}
