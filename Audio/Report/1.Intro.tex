\section{Introduction}

Source separation is a well-known research field in audio processing.
It aims at recovering original signals on a mixture made of them.
In addition to the difficulty of the task itself, research in source separation
 is also motivated for its applications in automatic
speech recognition, and fundamental frequency estimation.
This research field has different setup. For example, in the case of blind
signal separation,~\cite{comon2010handbook} the algorithm does not know the number of sources.

In this report, we focus on the case of music source separation.
Observing a mixture of several audio signals $s_j$ at the discrete time index $n$:
$$x(n) = \sum_{j\in \mathcal{J}} {s_j(n)},$$
the mission consists of finding estimates $\hat{s}_j$ of $s_j$.
Recent papers investigated the use of deep neural networks (DNN),
 in order to regress amplitude features of each estimated signals $\hat{s}_j$ using  short-therm audio samples.

While the ideas of using dense or recurrent neural networks for music-based applications were
already proposed  by Todd \cite{Todd1988} and Lewis \cite{Lewis1988} in 1988,
music source separation became particularly popular from 2014
thanks to challenges proposed by MIREX and SiSEC.

The originality of Muth et al.'s approach \cite{muth2018improving} is to
successfully employ the phase as in an input feature in the prediction of
the amplitude of each estimated signal $\hat{s}_j$.
In particular, they show that an adequate pre-processing step allows
them to enhance state-of-the-art results on the DSD100 dataset.

The DSD100 dataset is composed of 100 songs split half
between the training set and the testing set. For each song, the mixture signal
and the original source signals (vocals, drums, bass, and other) are given.

The first Section of this report dedicates itself to the use a DNN-based approach
 for separating musical sources. Then, experiments are presented,
  which lead to an analysis of the approach's limit.  
Finally, enhancements inspired by recent papers are proposed.
